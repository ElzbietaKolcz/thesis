\documentclass[12pt]{report}
\usepackage{graphicx}
\usepackage{blindtext}
\usepackage{caption}
\usepackage{subcaption}
\usepackage{titlesec}
\usepackage{lipsum}
\usepackage{tabu}
\usepackage[inkscapeformat=png]{svg}
\usepackage[polish]{babel}
\usepackage[T1]{fontenc}
\usepackage[utf8]{inputenc}
\usepackage{csquotes}
\usepackage[backend=biber]{biblatex}
\usepackage[a4paper,width=150mm,top=25mm,bottom=25mm,bindingoffset=6mm]{geometry}

\addbibresource{bibliography.bib}
\pagestyle{fancy}
\graphicspath{ {images/} }
\linespread{1.5}
\title{Title goes here}
\author{Name&Surname}
\date{Day Month Year}

\renewcommand*\contentsname{Spis treści}
\renewcommand{\listfigurename}{Spis rysunków}
\renewcommand{\listtablename}{Spis tabel}
\titleformat{\chapter}[display]
  {\normalfont\huge\bfseries}{\chaptertitlename\ \thechapter}{20pt}{\Huge}
\titlespacing*{\chapter}{0pt}{-50pt}{20pt}

\begin{document}
% !TeX spellcheck = pl
\begin{titlepage}
    \begin{center}
        \vspace*{1cm}
        
        \Huge
        \textbf{ZodiaCal}
        
        \vspace{0.5cm}
        \LARGE
        Projekt mobilnego kalendarza

        \vspace{5cm}

        
        \textbf{Elżbieta Kołcz 81640}


        \vspace{0.5cm}

        \Large
        Uniwersytet WSB Merito we Wrocławiu\\
        Wydział Finansów i Zarządzania\\
        Kierunek: Informatyka inżynierska\\
        2025
        
        \vfill
        Opiekun naukowy\\
        dr inż. Stanisław Lota

        
        \vspace{0.8cm}
    \end{center}
\end{titlepage}

\tableofcontents

\listoffigures

\chapter{Wprowadzenie}
\section{Główne założenia}
Napisać o natywnych aplikacjach \\

\phantom{Th} 
ZodiaCal to aplikacja kalendarz, która umożliwia użytkownikom łatwą organizację czasu poprzez przejrzysty i intuicyjny design. Jest to produkt skierowany głównie do osób jednocześnie pracujących jak i studiujących, ponieważ posiada opcję automatycznego uzupełniania pól w kalendarzu z rozróżnieniem na kategorie praca / studia.\\


Aplikacja ta również jest pomocna podczas złożonej pielęgnacji cery, gdyż po miesiącu korzystania z produktu, aplikacja na podstawie wcześniejszych miesięcy analizuje dane z pielęgnacji i wskazuje w jakim stopniu pielęgnacja z danego miesiąca była na przykład  nawilżająca, złuszczająca czy regeneracyjna. \\ 


Dodatkowo korzystając z aplikacji, użytkownicy mają dostęp do codziennych horoskopów dostosowane do ich znaku zodiaku. 


\chapter{Analiza i badania rynku}
\section{Analiza konkurencyjnych rozwiązań}

\phantom{Th}

Na rynku znajduje się wiele narzędzi do organizacji czasu, ale \textbf{kalendarz Google} pozostaje niekwestionowanym liderem.
Jest to profesjonalne i przede wszystkim darmowe, preinstalowane narzędzie na większości urządzeń z systemem Android.
Według raportu stworzonego przez zespół pracujący dla DataReportal, w Polsce na rok 2023, aż 87\% użytkowników posiada
telefon właśnie z tym systemem
\cite{datareportal}. Aplikacja ta stanowi nieodłączny element funkcjonowania firm, zwłaszcza tych początkujących,
ale nie tylko. Ułatwia harmonogramowanie spotkań, śledzenie projektów i pozwala na zachowanie kontroli nad terminami.
Dużym atutem kalendarza Google jest rozdzielenie zadań i wydarzeń na osobne elementy kalendarza (Rys. \ref{fig:googleCalendar}).\\

\begin{figure}[ht]
  \begin{minipage}{0.4\textwidth}
    \centering
    \includegraphics[height=13cm, keepaspectratio]{images/analiza/googleCalendar}
    \caption{Widok konkretnego dnia w kalendarzu Google}
    \label{fig:googleCalendar}
  \end{minipage}
  \hfill
  \begin{minipage}{0.4\textwidth}
    \centering
    \includegraphics[height=13cm, keepaspectratio]{images/analiza/protonCalendar}
    \caption{Widok tworzenia wydarzenia w kalendarzu Proton}
    \label{fig:protonCalendar}
  \end{minipage}
\end{figure}

Jest to narzędzie umożliwiające precyzyjne określenie wszelkich informacji odnośnie danego wydarzenia.
Nie da się kwestionować, że możliwości jakie daje nam aplikacja są niepotrzebne,
jednakże z punktu widzenia użytkownika końcowego ZodiaCal funkcje takie jak lokalizacja wydarzenia,
link do rozmowy wideo, czy określanie trwania każdego jednego wpisanego wydarzenia od do jest zbędne.
W ZodiaCal istotne jest sprawne dodawanie wydarzeń do kalendarza, bez konieczności przeglądania dodatkowych opcji.\\

Dobrą alternatywą dla kalendarza Google jest \textbf{Proton Calendar} (Rys. \ref{fig:protonCalendar}).
Jest to również zaawansowana aplikacja do monitorowania wydarzeń, z tą różnicą,
iż proton specjalizuje się w zapewnianiu jeszcze większej prywatności i bezpieczeństwa użytkowników.
Z tym, że nie rozróżnia wydarzeń od zadań. Proton to firma oferująca usługi związane
z prywatnością online, w tym bezpiecznymi skrzynkami e-mail i kalendarzami.\\


Bardzo ciekawym i minimalistycznym rozwiązaniem jest aplikacja \textbf{135 To Do List} która poprzez swój estetyczny
i prosty interfejs ułatwia użytkownikom priorytetyzacja zadań na konkretny dzień.
Posiada możliwość zmiany kolejności wpisanych już wcześniej zadań oraz widok całego miesiąca.
Niestety nie ma opcji dodawania cyklicznych wydarzeń. ZodiaCal przyświeca niemalże ta sama minimalistyczna
idea tworzenia interfejsu, jednak wprowadza rozbudowane funkcje, takie jak codzienny horoskop i osobisty dziennik pielęgnacji (Rys. \ref{fig:ToDoList}).\\

\textbf{FeelingMySkin} to bardzo rozbudowane narzędzie do precyzyjnego określenia pielęgnacji cery i nie tylko.
Korzystając z niej użytkownik może układać plany pielęgnacyjne z wykorzystaniem konkretnych produktów,
których używa na co dzień,  monitorować zmiany skórne, czy chociażby śledzić daty przydatności produktów.
To z pewnością przydatna aplikacja, jednakże wymaga czasu, aby opanować wszystkie możliwe funkcje.
Interfejs strony głównej jest bardzo przeładowany informacjami, co utrudnia korzystanie z aplikacji w sposób efektywny (Rys. \ref{fig:feelingMySkin}).\\

\begin{figure}[ht]
  \begin{minipage}{0.4\textwidth}
    \centering
    \includegraphics[height=13cm, keepaspectratio]{images/analiza/135ToDoList}
    \caption{Ekran główny aplikacji 135 To Do List}
    \label{fig:ToDoList}
  \end{minipage}
  \hfill
  \begin{minipage}{0.4\textwidth}
    \centering
    \includegraphics[height=13cm, keepaspectratio]{images/analiza/feelingMySkin}
    \caption{Widok codziennej rutyny w aplikacji FeelingMySkin}
    \label{fig:feelingMySkin}
  \end{minipage}
\end{figure}

ZodiaCal to hybryda wszystkich wymienionych aplikacji.
To minimalistyczne narzędzie służące do wielu celów, ale w podstawowym zakresie.
Próbuje połączyć najlepsze cechy różnych istniejących aplikacji,
oferując minimalistyczny interfejs do zarządzania czasem, zadaniami i nie tylko.
Chociaż nie zawiera wszystkich zaawansowanych funkcji dostępnych w innych aplikacjach,
dąży do dostarczenia użytkownikom wyważonego narzędzia do pracy.


\section{Analiza SWOT}
Napisać tutaj coś o analizie SWOT

\phantom{Th}

\textbf{Mocne strony (Strengths):}\\
•	Prostota: Aplikacja łączy w sobie wyłącznie najpotrzebniejsze funkcje typowego, biurowego kalendarza i dzienniczka pielęgnacji, przez co jest bardziej przystępna dla początkujących.\\
•	Spersonalizowana pielęgnacja: Możliwość dostosowywania rutyn pielęgnacyjnych do konkretnych potrzeb użytkownika.\\
•	Powiadomienia i przypomnienia: Ułatwia utrzymanie regularności w dbaniu o cerę.\\
•	Uczenie maszynowe: Propozycje pielęgnacji na podstawie wcześniejszych danych.\\
•	Wieloplatformowość: Aplikacja działa na wielu platformach dzięki czemu można dotrzeć do szerszej grupy użytkowników.\\

\textbf{Słabe strony (Weaknesses):}\\
•	Brak bazy danych produktów: W aplikacji nie ma możliwości zaciągnięcia z bazy danych kosmetyków, których użytkownik używa podczas pielęgnacji.\\
•	Brak synchronizacji z pocztą: Jeśli użytkowni otrzyma maila odnośnie jakiegoś wydarzenia, nie doda się ono automatycznie do kalendarza.\\

\textbf{Szanse (Opportunities):}\\
•	Rynek pielęgnacji cery: W ostatnich czasach zainteresowanie świadomą pielęgnacją skóry wzrosło co stwarza możliwość przyciągnięcia nowych użytkowników.\\
•	Wprowadzenie płatnej wersji z dodatkowymi funkcjami: Możliwość wprowadzenia płatnej wersji aplikacji z rozszerzonymi funkcjami dla zaawansowanych użytkowników.\\

\textbf{Zagrożenia (Threats):}\\
•	Prywatność i bezpieczeństwo danych: Konieczność zapewnienia wysokiego poziomu ochrony danych użytkowników, zwłaszcza w obszarze związanym z informacjami o pielęgnacji skóry.\\
•	Zmiany w regulacjach dotyczących prywatności: Zmiany przepisów dotyczących ochrony danych osobowych mogą wpłynąć na wymogi dotyczące zachowania prywatności użytkowników i mogą wymagać dostosowania polityki prywatności aplikacji.\\

\section{Analiza przedmiotowa MoSCoW}

MUST
•	Funkcja logowania i rejestracji – połączenie z bazą danych
•	Funkcjonalności związane z kalendarzem 
o	CRUD wydarzeń 
o	Widok roczny
o	Widok miesięczny
o	Widok tygodniowy 
•	Automatyczne uzupełnianie pól cyklicznych wydarzeń w kalendarzu praca/studia
•	Funkcja przydzielania znaku zodiaku z podanej daty przez użytkownika podczas rejestracji 
•	Dzienny horoskop – połączenie z API (RTK-query)
SHOULD
•	Funkcjonalność związana z dodawaniem produktów do danej pielęgnacji 
•	Funkcjonalność związana z zarejestrowaniem danej pielęgnacji na dzień
•	Formularz zgłaszania błędów w aplikacji 
•	Logowanie kontem Google
COULD
•	Analiza pielęgnacji – wykresy 
•	Onboarding zaraz po zarejestrowaniu konta 
•	Powiadomienia 
•	Traker samopoczucia, snu i nawyków
•	Możliwość zmiany języka


WON’T
•	Personalizacja motywów (kolory interfejsu)
•	Wersja na zegarek – możliwość nadzoru pielęgnacji 
•	Propozycja pielęgnacji po 3 miesiącach korzystania aplikacji 


\chapter{Technologia}
Napisać o systemie kontroli wersji, co to jest Api, o Expo
Korzystać z dokumentacji!!\\

W poniższym rozdziale zostały przedstawione kluczowe technologie potrzebne do wykonania aplikacji.\\
\phantom{Th} 

\section{React Native}\\


\cite{reactnative} React Native to technologia oparta na bibliotece React, która jest używana do tworzenia interfejsów za pomocą języka JavaScript.  Biblioteka ta jest typu open source, co oznacza, że jej kod źródłowy jest udostępniony do użytku publicznego i każdy ma prawo go modyfikować według własnych potrzeb. \\
 
React Native służy do tworzenia wieloplatformowych aplikacji, znanych również jako aplikacje cross-platform, czyli aplikacji zaprojektowanych i napisanych w taki sposób, aby działały na wielu różnych platformach i systemach operacyjnych. Dodatkowo wykorzystuje programowanie natywne zapewniając w ten sposób zestaw komponentów niezależnych od systemu operacyjnego, który następnie jest dynamicznie przekształcany na natywne elementy interfejsu danego systemu. Dzięki temu aplikacja jest w stanie zachować spójność i dostosować się do  konkretnej specyfiki.\\

Wielką zaletą React Native jest jego szybkie odświeżanie, co przyczynia się do oszczędności czasu podczas pisania kodu. Jest to możliwe dzięki wykorzystaniu JavaScript przez React Native.\\


\section{NativeWind}\\

\cite{nativewind} NativeWind wykorzystuje Tailwind CSS jako język skryptowy do tworzenia uniwersalnego systemu styli dla React Native. Z kolei Tailwind CSS to framework do tworzenia interfejsów użytkownika, który opiera się na klasach CSS, ułatwiając tym samym stylizację aplikacji. Framework w tym kontekście to rodzaj struktury lub szkieletu, który zapewnia gotowe rozwiązania, abstrakcje i narzędzia do rozwoju interfejsu. Tailwind CSS dostarcza zestaw predefiniowanych klas CSS, które reprezentują różne style, kolory, marginesy, wypełnienia dzięki czemu interfejs jest konsekwentny i spójny.  \\

Choć React Native nie obsługuje tradycyjnych stylów CSS, narzędzie NativeWind pozwala na przekształcanie klas CSS zapisanych w stylach Tailwind na odpowiednie style i komponenty dostępne w React Native, umożliwiając wykorzystanie tych klas w aplikacjach mobilnych.\\

\section{Expo Go}\\
\cite{expogo} Expo to platforma typu open source dla aplikacji działających natywnie na różnych systemach operacyjnych. Expo umożliwia testowania aplikacji na różnych urządzeniach mobilnych. Aby korzystać z Expo wymagane jest pisanie kodu w JavaScript/TypeScript oraz posiadanie aplikacji na urządeniu mobilnym, które można pobrać z Sklepu Play w przypadku Androida lub z App Store w przypadku iOS'a\\


\section{Google Firebase}\\
Choć React Native nie obsługuje tradycyjnych stylów CSS, narzędzie NativeWind pozwala na przekształcanie klas CSS zapisanych w stylach Tailwind na odpowiednie style i komponenty dostępne w React Native, umożliwiając wykorzystanie tych klas w aplikacjach mobilnych.\\


\section{Postman}\\
Choć React Native nie obsługuje tradycyjnych stylów CSS, narzędzie NativeWind pozwala na przekształcanie klas CSS zapisanych w stylach Tailwind na odpowiednie style i komponenty dostępne w React Native, umożliwiając wykorzystanie tych klas w aplikacjach mobilnych.\\


\section{Visual Studio Code}\\
Choć React Native nie obsługuje tradycyjnych stylów CSS, narzędzie NativeWind pozwala na przekształcanie klas CSS zapisanych w stylach Tailwind na odpowiednie style i komponenty dostępne w React Native, umożliwiając wykorzystanie tych klas w aplikacjach mobilnych.\\


\section{Github}\\
Choć React Native nie obsługuje tradycyjnych stylów CSS, narzędzie NativeWind pozwala na przekształcanie klas CSS zapisanych w stylach Tailwind na odpowiednie style i komponenty dostępne w React Native, umożliwiając wykorzystanie tych klas w aplikacjach mobilnych.\\


\section{Figma}\\
Choć React Native nie obsługuje tradycyjnych stylów CSS, narzędzie NativeWind pozwala na przekształcanie klas CSS zapisanych w stylach Tailwind na odpowiednie style i komponenty dostępne w React Native, umożliwiając wykorzystanie tych klas w aplikacjach mobilnych.\\



\chapter{Wymagania aplikacji}
\section{Analiza przedmiotowa MoSCoW}
\phantom{Th}
Analiza MoSCoW podobnie jak analiza SWOT jest akronimem. Dzieli wymagania na cztery kategorię, uwzględniające wzrastającą złożoność.
Są to kolejno M (Must have) wymagania konieczne, S (Should have) wymagania wskazane, C (Could have) wymagania opcjonalne,
W (Won't have) wymagania wykluczone. Metoda ta pomaga w ustaleniu priorytetów, które w sposób uzasadniony zapewnia kompletny
i dokładny zestaw wymagań, które aplikacja musi spełnić. Takie podejście uświadamia, które funkcje są niezbędne do funkcjonowania systemu
i zapobiega realizacji zadań opcjonalnych przed ukończeniem zadań obowiązkowych. W rezultacie umożliwia bardziej efektywne zarządzanie projektem i zaspokojenie kluczowych potrzeb użytkowników \cite{moscow},\cite{businessanalysis}.

\subsubsection*{\textbf{Must have}}
\phantom{Th}
Zbiór funkcji oznaczony jako Must have to kategorie wymagań, które system jest zobowiązany posiadać.
Podstawowym elementem aplikacji jest połączenie z bazą danych, którą należy w pierwszej kolejności skonfigurować
i podłączyć. Konieczne jest utworzenie modelu danych oraz zapewnienie niezbędnych operacji,
które umożliwią efektywne zarządzanie danymi użytkowników w systemie. Aby korzystać z aplikacji potrzebne jest konto,
co oznacza, że niezbędnym elementem jest ekran logowania i rejestracji.
Ponadto ekrany te muszą posiadać walidację formularzy przed wysłaniem ich na serwer. Zapewni to poprawność
i integralność danych wprowadzanych przez użytkowników oraz pomoże uniknąć potencjalnych błędów
i problemów związanych z danymi przechowywanymi w systemie.

Drugim fundamentem aplikacji jest funkcja kalendarza. Musi ona zapewniać możliwość dodawania, usuwania oraz edytowania wydarzeń, zadań.
Dodatkowo ważnym aspektem jest widok kalendarza na cały rok, miesiąc, tydzień i dzień.
Niezbędną funkcją jest również możliwość rejestrowania swojej pielęgnacji na dany dzień, w zależności od kategorii nawilżanie,
złuszczanie, odbudowa, przerwa. Kolejną kluczową funkcją jest wyświetlanie horoskopu dla znaku zodiaku użytkownika,
co wiąże się ze znalezieniem odpowiedniego API oraz połączenia go z aplikacją.

\subsubsection*{\textbf{Should have}}
\phantom{Th}
W kontekście sekcji Should have skupiamy się na wymaganiach, które są ważne, ale nie są absolutnie niezbędne dla funkcjonowania systemu.
Przydatną funkcją związaną z kalendarzem, z perspektywy użytkownika końcowego jest możliwość automatycznego uzupełnianie pól cyklicznych
takich jak praca/studia oraz otrzymywanie powiadomień. Z kolei rozszerzeniem funkcji pielęgnacji jest opcja związana z dodawaniem
produktów do danej pielęgnacji. Natomiast rozbudowaniem modułu horoskopu jest funkcja przydzielania znaku zodiaku
z podanej daty przez użytkownika. Wiele aplikacji umożliwia logowanie się za pomocą kont z mediów społecznościowych, dlatego warto rozważyć dodanie opcji logowania przy użyciu konta Google, aby sprostać obowiązującym standardom. To umożliwi rozszerzenie funkcjonalności systemu, dostarczając użytkownikom dodatkowych wygodnych i atrakcyjnych opcji.

\subsubsection*{\textbf{Could have}}
\phantom{Th}
Funkcje w grupie Could have to wymagania, które dobrze jest mieć, ale nie są kluczowe.
Dobrym obszarem do rozbudowy jest moduł związany z pielęgnacją.
Można rozszerzyć go o wyświetlaniu wykresów pod koniec miesiąca,
monitorowanie daty ważności używanych produktów oraz dodanie Trackerów samopoczucia, snu i nawyków.
Tracker to narzędzie pozwalające na odnotowanie aktywności w danym stopniu na konkretny dzień.
Na pewno dużym ułatwieniem dal użytkownika byłoby wprowadzenie możliwości zmiany języka oraz onboardingu zaraz po zarejestrowaniu konta.

\subsubsection*{\textbf{Won't have}}
\phantom{Th}
Wymagania w grupie Won't have mają najniższy priorytet. Określają funkcję, które w tej wersji nie zostaną dostarczone,
ale będą zawarte w kolejnej aktualizacji. Klienci często lubią edytować podstawowy interfejs pod siebie,
więc wprowadzenie personalizacji motywów na pewno korzystnie wpłynęłaby na odbiór aplikacji.
Dodatkowo coraz więcej użytkowników decyduje się na smartwatche.
Przyszłościowym krokiem byłoby umożliwienie korzystania z aplikacji również na zegarku.

\section{Wymagania funkcjonalne:}
\phantom{Th}
Podstawowymi funkcjonalnościami jest wszystko to co jest związane z obsługą
kalendarza, czyli dodawanie, usuwanie oraz edycja wydarzeń, zadań, celów czy harmonogramu
na dany dzień, miesiąc. Dodatkowo konieczne jest, aby użytkownik mógł zobaczyć widok
kalendarza na dany miesiąc, tydzień oraz rok. ZodiaCal to nie tylko kalendarz, ale również
dzienniczek pielęgnacji cery. Stąd potrzeba uwzględnienia możliwości wyboru typu pielęgnacji,
dodania, usunięcia, edytowania produktów do danej pielęgnacji oraz funkcja pozwalająca
zarejestrować pielęgnację na dany dzień. Ponadto aplikacja dostarcza codziennie
użytkownikowi horoskop na dany dzień uwzględniając podany wcześniej przez niego znak
zodiaku, lub datę urodzenia.

\section{Wymagania niefunkcjonalne:}
\phantom{Th}
Aby korzystać z aplikacji ZodiaCal, konieczne jest posiadanie telefonu z systemem
Android w wersji 5.0 lub nowszej, lub systemem iOS w wersji 13 lub nowszej. Ponieważ
aplikacja pobiera dane w czasie rzeczywistym i wykorzystuje podstawowy pakiet Firebase,
należy wziąć pod uwagę, że maksymalnie 100 urządzeń może jednocześnie synchronizować się
z bazą danych. Dodatkowo, maksymalna przepustowość danych jest zazwyczaj ograniczona do
360 MB na dzień, co reguluje polityka ograniczeń Firebase \cite{limits}.

\chapter{Projekt interfejsu graficznego}
\addcontentsline{toc}{chapter}{Projekt interfejsu graficznego}

Dzisiejsze interfejsy graficzne są projektowane zgodnie zasadami User Experience oraz User Interface. Są to zbiory wytycznych dla projektantów skoncentrowane na doświadczeniach użytkownika końcowego. User Experience to dziedzina zajmująca się badaniem potrzeb użytkownika, natomiast User Interface odpowiada za szatę graficzną produktu \cite{uxui}. ZodiaCal zostało stworzone zgodnie z powyższymi zasadami. Poniżej znajdują się Wireframes, czyli modele szkieletowe, zwane również jako architektura widoku aplikacji przedstawiające projekt graficzny poszczególnych ekranów wykonem w aplikacji internetowej Figma. Są to High-Fidelity Wireframes, co oznacza, że stanowią dopracowane rozwiązanie z szczegółowymi detalami \cite{uxui}.


\begin{figure}[h]
	\begin{minipage}{0.3\textwidth}
		\centering
		\includegraphics[height=4cm, keepaspectratio]{images/logo/logo_text}
		\caption{Logo wraz z typografią}
		\label{fig:logo_text}
	\end{minipage}
	\hfill
	\begin{minipage}{0.3\textwidth}
		\centering
		\includegraphics[height=4cm, keepaspectratio]{images/logo/logo}
		\caption{Logo bez typografii}
		\label{fig:logo}
	\end{minipage}
	\hfill
	\begin{minipage}{0.3\textwidth}
		\centering
		\includegraphics[height=4cm, keepaspectratio]{images/logo/favicon}
		\caption{Ikona aplikacji favicon}
		\label{fig:favicon}
	\end{minipage}
\end{figure}

Rysunek \ref{fig:logo_text} przedstawia logo aplikacji ZodiaCal. Zostało one zaprojektowane z wykorzystaniem narzędzia Figma, który umożliwia tworzenie obrazów wektorowych, to znaczy obrazów opisanych za pomocą kształtów, linii i krawędzi w odróżnieniu do grafiki rastrowej opartej na pikselach. Dzięki wykorzystaniu wektorów grafiki nie tracą swojej ostrości
podczas powiększani, czy zmieniania obiektu. Stworzone logo miała nawiązywać do astrologicznych motywów tak, aby przywodziło na myśl horoskopy oraz wróżbiarstwo. Font wykorzystany do nazwy aplikacji został staranie wybrany, aby oddać charakter aplikacji. Jest to
font szeryfowy posiadający małe dekoracyjne elementy. Wykorzystanie takie rodzaju typografii ma na celu wywołanie wrażenia, że aplikacja prezentuje się w sposób klasyczny i elegancki.


\begin{figure}[t]
	\begin{minipage}{0.4\textwidth}
		\centering
		\includegraphics[height=10cm, keepaspectratio]{images/interfejs_figma/Sign_in}
		\caption{Ekran logowania}
		\label{fig:Sign-in}
	\end{minipage}
	\hfill
	\begin{minipage}{0.4\textwidth}
		\centering
		\includegraphics[height=10cm, keepaspectratio]{images/interfejs_figma/Sign_up}
		\caption{Ekran rejestracji}
		\label{fig:Sign-up}
	\end{minipage}
\end{figure}

\section*{Projekt ekranu logowania}
\addcontentsline{toc}{section}{Projekt ekranu logowania}
Ekran logowania (Rys. \ref{fig:Sign-in}) przedstawia minimalistyczny formularz logowania do aplikacji, który zawiera dwa pola umożliwiające wprowadzeniu tekstu typu TextInput. TextInputy są interaktywne, to znaczy, że reagują na dane wpisane przez użytkownika. Takie rozwiązanie pozwala na komunikację klienta z aplikacją w celu uniknięcia błędów podczas wpisywania tekstu oraz frustracji użytkownika, który bez komunikatu nie wiedziałby co się dzieje na ekranie. Pole z hasłem ma ikonę oczka, dzięki której można podejrzeć wpisane hasło. Dodatkowo użytkownik ma możliwość zalogowania się do aplikacji poprzez serwisy społecznościowe takie jak konto Google, czy konto Apple. Jeśli klient nie ma jeszcze zarejestrowanego konta może przejść do ekranu rejestracji klikając w przycisk „Sign up”.

\section*{Projekt ekranu rejestracji}
\addcontentsline{toc}{section}{Projekt ekranu rejestracji}
Rejestracja użytkownika (Rys. \ref{fig:Sign-up}) odbywa się poprzez uzupełnienie trzech pól w formularzu. Podobnie jak w przypadku ekranu logowania pole „Password” posiada ikonę oczka oraz możliwe jest utworzenie konta z wykorzystaniem innych kont społecznościowych. Po poprawnym wprowadzeniu danych i kliknięciu przycisku „Sign up”, użytkownik zostaje
przeniesiony do ekranu głównego aplikacji.

\begin{figure}[t]
	\begin{center}
		\begin{minipage}{0.4\textwidth}
			\centering
			\includegraphics[height=10cm, keepaspectratio]{images/interfejs_figma/Home}
			\caption{Ekran główny}
			\label{fig:Home}
		\end{minipage}
	\end{center}
\end{figure}

\section*{Projekt ekranu głównego}
\addcontentsline{toc}{section}{Projekt ekranu głównego}
Po udanym zalogowaniu się do aplikacji, użytkownik jest powitany spersonalizowanym komunikatem, który zawiera jego imię (Rys. \ref{fig:Home}). Dodatkowo w tej samej płaszczyźnie znajduję się przycisk do wylogowania. Poniżej znajduję się kalendarz miesięczny z zaznaczonymi świętami, wydarzeniami oraz urodzinami użytkownika. Wygubiony numer w kolorze fioletowym reprezentuje obecny dzień, numer w żółtym kole oznacza święto, numer w fioletowym kole informuje o urodzinach bliskich, mała czerwona kropeczka pod numerem wskazuje na nadchodzące wydarzenie. Pod kalendarzem zostały umieszczone wszystkie informację o zaznaczonych polach w kalendarzu. Dodatkowo poniżej znajduje się sekcja „Goals for this month” w której użytkownik może wprowadzić trzy cele, które chciałby osiągnąć w danym miesiącu. Każde pole posiada checkbox, który informuje, czy klient ukończył cel, jeśli tak checkbox zostaje wypełniony kolorem fioletowym, a tekst w polu zostaje przekreślony, opcja edytowania tekstu zostaje wyłączona. Po prawej stronie znajdują się okrągłe przyciski. Jeśli pole jest puste, przyciski reprezentuje plusik, jeśli w polu już cos się znajduje ikona zmienia się na ołówek, co oznacza, że pole można edytować. W całej aplikacji została wprowadzona nawigacja dolna. Zawiera ona ikony i nazwy odzwierciedlające dostępne ekrany. Zawsze aktualny ekran jest wyróżniony w pasku nawigacyjnym.

\begin{figure}[t]
	\begin{minipage}{0.4\textwidth}
		\centering
		\includegraphics[height=10cm, keepaspectratio]{images/interfejs_figma/Horoscope-onboarding}
		\caption{Onboarding horoskopu}
		\label{fig:Onboarding}
	\end{minipage}
	\hfill
	\begin{minipage}{0.4\textwidth}
		\centering
		\includegraphics[height=10cm,           keepaspectratio]{images/interfejs_figma/Horoscope}
		\caption{Ekran dziennego horoskopu}
		\label{fig:Horoscope}
	\end{minipage}
\end{figure}

\section*{Projekt ekranu horoskopu}
\addcontentsline{toc}{section}{Projekt ekranu horoskopu}
Pierwszą zakładką w nawigacji jest ekran horoskopu, który może wyglądać na dwa sposoby w zależności od tego, czy użytkownik wprowadził swój znak zodiaku do systemu.

Pierwszy scenariusz został ukazany na rysunku \ref{fig:Onboarding}. Jeśli klient uruchamia tę zakładkę po raz pierwszy, jego oczom ukaże się logo aplikacji oraz dwa małe formularze. Użytkownik, jeśli zna swój znak zodiaku może go wprowadzić w pierwszym formularzu. Jeśli jednak dopiero rozpoczyna swoją przygodę z horoskopami, aplikacja pomoże mu określić swój znak zodiaku. Wystarczy podać dzień i miesiąc urodzenia, a program sam przydzieli znak zodiaku i wyświetli go na ekranie wraz z horoskopem.

Rysunek \ref{fig:Horoscope} przedstawia sytuację, gdzie w systemie już znajduję się informacja o znaku zodiaku użytkownika. Logo aplikacji zostaje zamienione na ilustrację przetrawiającą wersję graficzną znaku zodiaku. Ilustracje wszystkich znaków zostały pobrane ze strony Freepik.com, który posiada darmową licencję do korzystania z zasobów znajdujących się na stronie w zakresie niekomercyjnym. Każdego dnia na tym ekranie pojawia się inna wróżba zodiaku.

\begin{figure}[t]
	\begin{minipage}{0.4\textwidth}
		\centering
		\includegraphics[height=10cm, keepaspectratio]{images/interfejs_figma/Year}
		\caption{Widok kalendarza rocznego}
		\label{fig:Year}
	\end{minipage}
	\hfill
	\begin{minipage}{0.4\textwidth}
		\centering
		\includegraphics[height=10cm,           keepaspectratio]{images/interfejs_figma/Birthday-Edit}
		\caption{Ekran listy dat urodzin}
		\label{fig:Birthday}
	\end{minipage}
\end{figure}

\section*{Projekt widoku kalendarza rocznego}
\addcontentsline{toc}{section}{Projekt widoku kalendarza rocznego}
Kolejna zakładka o nazwie „Year” przenosi użytkownika do widoku kalendarza rocznego \ref{fig:Year}). W tym ekranie wyświetlane są kalendarze miesięcy jedynie danego roku. Dzięki takiemu rozwiązaniu klient nie musi się przejmować zbędnym skorolowaniem, które może omyłkowo zaprowadzić go do zupełnie innego roku. W tym widoku kalendarza zaznaczone są jedynie święta oraz urodziny. Poniżej każdego kalendarza na dany miesiąc znajduję się informacja o zaznaczonych polach. Na górze ekranu został umieszczony formularz do dodawania urodziny do listy użytkownika. Zawiera on trzy pola: day, month oraz name. Po prawej stronie znajduje się przycisk dodawania urodzin do listy. Jeśli użytkownik się pomyli lub
po prosu będzie chciał edytować listę z datami, musi kliknąć w okrągły przycisk z ikoną ołówka. Otworzy się wtedy tak zwany portal (Rys. \ref{fig:Birthday}) przedstawiający tabelę z czterema kolumnami: day, month, name i delete. To właśnie tam użytkownik może usunąć złe rekordy. Tabela sortuję się w taki sposób, aby na początku zawsze były pola z poprawną kolejnością miesięcy, a następnie w sposób rosnący reprezentuje kolejność dni w miesiącu.

\begin{figure}[t]
	\begin{minipage}{0.4\textwidth}
		\centering
		\includegraphics[height=10cm, keepaspectratio]{images/interfejs_figma/Week}
		\caption{Widok kalendarza tygodniowego}
		\label{fig:Week}
	\end{minipage}
	\hfill
	\begin{minipage}{0.4\textwidth}
		\centering
		\includegraphics[height=10cm,           keepaspectratio]{images/interfejs_figma/Task_Event}
		\caption{Ekran dodawania zadań i wydarzeń do kalendarza}
		\label{fig:Task_Event}
	\end{minipage}
\end{figure}

\section*{Projekt widoku kalendarza tygodniowego}
\addcontentsline{toc}{section}{Projekt widoku kalendarza tygodniowego}
Rysunek \ref{fig:Week} reprezentuje ekran zakładki „Week”. W tym widoku  użytkownik może podglądać swoje zadania i wydarzenia na konkretny tydzień. Na samej górze znajduje się kalendarz z pogrubionym obecnym dniem, zaznaczonym świętem i wydarzeniem. Poniżej dla każdego dnia zostają wyświetlone zdania i wydarzenia, które użytkownik wprowadził na dany dzień. Klient może edytować pola zadań i wydarzeń poprzez kliknięcie ikony ołówka. Inputy mają takie samo zachowanie jak w przypadku komponentu Goals w ekranie głównym. Na samym dole znajduje się miejsce na wpisanie przez użytkownika celów na ten tydzień.

Aby dodać zadanie lub wydarzenie należy kliknąć okrągły przycisk z plusem. Po tej akcji pojawi się portal z formularzem do tworzenia zadań i wydarzeń (Rys. \ref{fig:Task_Event}). Jest to bardzo minimalistyczny interfejs zawierający dwa TextInput day i name oraz przycisk dodawania. Aby zachować spójność projektu poniżej każdego z formularzy znajduje się tabela z trzema kolumnami: day, name, delete. Z tego miejsca użytkowni może zarządzać swoimi zadaniami oraz wydarzeniami na dany tydzień. Tabela sortuje rekordy w sposób rosnący.

\begin{figure}[t]
	\begin{center}
		\begin{minipage}{0.4\textwidth}
			\centering
			\includegraphics[height=10cm, keepaspectratio]{images/interfejs_figma/SkinCare}
			\caption{Ekran pielęgnacji na dany dzień}
			\label{fig:SkinCare}
		\end{minipage}
	\end{center}
\end{figure}

\section*{Projekt ekranu pielęgnacji}
\addcontentsline{toc}{section}{Projekt ekranu pielęgnacji}
Ostatnią zakładką w pasku nawigacyjnym jest SkinCare (Rys. \ref{fig:SkinCare}). Podobnie jak w ekranie widoku kalendarza tygodniowego na górze znajduje się kalendarz tygodniowy, z tym że na nim nie ma już zaznaczonych żadnych świąt, wydarzeń czy zadań. Dodatkowo klikając w poszczególne dni na kalendarzu poniżej wyświetla się pielęgnacja dedykowana na konkretny dzień, w ekranie widoku kalendarza tygodniowego, wszystkie zadania, wydarzenia, święta są wyświetlane razem. Pielęgnacja jest podzielona na dwie sekcje: pielęgnację poranną i pielęgnację wieczorną. Pielęgnacja na dzień zawsze zawiera te same produkty, chyba że użytkownik sam je zmieni, natomiast pielęgnacja na wieczór jest wybierana z wysuwanego menu z którego można wybrać jedną z czterech typów pielęgnacji: nawilżająca, złuszczająca, regeneracyjna lub przerwa. W zależności od tego jaką pielęgnacje użytkownik wybierze taka lista produktów zostanie wyświetlona w sekcji pielęgnacji wieczornej. Dodatkową funkcją jest możliwość kliknięcia przez użytkownika checkboxa, gdy wykorzystał produkt z wymienionej listy. Na samej górze znajduję się osobna mini nawigacja dla sekcji SkinCare. Aktualny widok jest zaznaczony kolorem fioletowym w nawigacji.

\begin{figure}[t]
	\begin{minipage}{0.4\textwidth}
		\centering
		\includegraphics[height=10cm, keepaspectratio]{images/interfejs_figma/SkinCare-Routines}
		\caption{Ekran tworzenia rutyny pielęgnacyjnej}
		\label{fig:Routines}
	\end{minipage}
	\hfill
	\begin{minipage}{0.4\textwidth}
		\centering
		\includegraphics[height=10cm,           keepaspectratio]{images/interfejs_figma/SkinCare-Summary}
		\caption{Ekran podsumowania miesiąca pielęgnacji}
		\label{fig:Summary}
	\end{minipage}
\end{figure}

Początkowo sekcja pielęgnacji porannej i wieczornej jest pusta, aby dodać kosmetyki należy kliknąć Rutines w górnej nawigacji. Po wykonaniu tego kroku użytkownikowi ukażą się formularze z nagłówkiem oraz polem do wpisania (Rys. \ref{fig:Routines}). Podczas wpisywania nazwy klient będzie mógł wybrać z bazy danych konkretne produkty. Po dodaniu produktu poniżej zostanie zaktualizowana tabela z listą produktów do danej pielęgnacji. W tabeli znajdują się cztery kolumny: name, brand, date oraz delete. Dla każdej sekcji jest osobny formularz oraz tabela. Ostatnim ekranem jest widok podsumowania pielęgnacji na dany miesiąc (Rys. \ref{fig:Summary}).

W tym miejscu są wyświetlane informacja w jakim stopniu pielęgnacja w tym miesiącu była nawilżająca, złuszczająca, regeneracyjna lub wystąpiła przerwa. Na środku znajduję się graf kołowy z wypisanymi procentami, a poniżej tabela z dwiema kolumnami: „Type of care” stanowiąca również legendę do wykresu oraz  „Number of occurrences”, która wyświetla ile razy dana pielęgnacja wystąpiła w miesiącu.



\chapter{Model danych}
\addcontentsline{toc}{chapter}{Model danych}
Diagramy UML zwane również Unified Modeling Language umożliwiają graficzne przedstawienie logiki relacji, czy procesu \cite{diagram}. Są to standardy graficzne używane w inżynierii oprogramowania do modelowania struktury i zachowania systemów informatycznych. Umożliwiają one lepsze zrozumienie architektury systemu oraz pomagają w procesie projektowania i dokumentowania. Poniżej zostały przedstawione diagramy kolekcji występujących w bazie danych Firebase wykorzystywanych w aplikacji ZodiaCal.

\section*{Diagram bazy danych}
\addcontentsline{toc}{section}{Diagram bazy danych}

\begin{figure}[h]
	\centering
	\includegraphics[width=1\linewidth]{images/model_danych/user}
	\caption{Diagram kolekcji user}
	\label{fig:user}
\end{figure}

Diagram kolekcji user (Tab. \ref{fig:user}) zawiera informację o zalogowanym użytkowniku. Każdy user posiada swój własny unikalny User UID. Podstawowe dane użytkownika to obowiązkowe pola email i user name oraz opcjonalne pole sign. Podczas korzystania z aplikacji w bazie danych tworzą się kolekcje takie jak obecny rok, w którym przechowywane są wszystkie informacje potrzebne do zarządzania kalendarzem, kolekcja birthday, w której użytkownik przechowuje daty urodzin swoich bliskich oraz kolekcję skinCare przechowujące treści dotyczące pielęgnacji cery.

\newpage

\begin{figure}[h]
	\centering
	\includegraphics[width=1\linewidth]{images/model_danych/birthday}
	\caption{Diagram kolekcji birthday}
	\label{fig:birthday}
\end{figure}

Kolekcja birthday (Tab. \ref{fig:birthday}) przedstawia listę dat urodzin różnych osób. Posiada dwa pola typu int tzn. day oraz month i jedno pole typu string name, które wskazuje na imię jubilata.

\begin{figure}[h]
	\centering
	\includegraphics[width=1\linewidth]{images/model_danych/skinCare}
	\caption{Diagram kolekcji skinCare}
	\label{fig:skinCare}
\end{figure}

W tabeli o nazwie "skinCare" (Tab. \ref{fig:skinCare}), znajduje się zbiór informacji dotyczących pielęgnacji cery. W ZodiaCal zostały wyróżnione 4 rodzaje pielęgnacji: złuszczanie, odbudowa, nawilżanie oraz przerwa. Każda kolekcja danego typu pielęgnacji posiada swoje własne produkty. Produkty natomiast zawierają informację o nazwie produktu, nazwie marki oraz terminie ważności kosmetyku. Użytkownik wybiera produkty do pielęgnacji z zewnętrznego API, a następnie są one przekazywane do Firebase.

\begin{figure}[h]
	\centering
	\includegraphics[width=1\linewidth]{images/model_danych/goals}
	\caption{Diagram kolekcji goals}
	\label{fig:goals}
\end{figure}

Kolekcja goals (Tab. \ref{fig:goals}) jest zagnieżdżona kolejno w kolekcji reprezentujący bieżący rok następnie bieżący miesiąc. W kolekcji odzwierciedlającej bieżący miesiąc znajdują się kolekcje dotyczące celi na dany miesiąc oraz tydzień, rutyny pielęgnacyjnej oraz zadań. Każdy cel ma swój unikalny ID, pole typu int przechowujące numer indexu, pole typu string name przedstawiające cel oraz pole typu bool state, które wskazuje czy cel został osiągnięty, czy nie.

\begin{figure}[h]
	\centering
	\includegraphics[width=1\linewidth]{images/model_danych/tasks}
	\caption{Diagram kolekcji tasks}
	\label{fig:tasks}
\end{figure}

Podobnie jak w przypadku kolekcji goals, tabela zatytułowana tasks (Tab. \ref{fig:tasks}) również jest zagnieżdżona w bieżącym roku oraz miesiącu. W tej kolekcji są tworzone kolejne kolekcje ilustrujące dni w danym miesiącu. Wszystkie zadania są przypisane do swoich unikalnych identyfikatorów, posiadają pola typu int, w których przechowywane są numery indeksów, pola typu string oznaczone jako "name" zawierające opis zadania, a także pola typu bool oznaczone jako "state", wskazujące na stan zadania.

\section*{Diagram UML komponentów}
\addcontentsline{toc}{section}{Diagram UML komponentów}

\begin{figure}[h]
	\centering
	\includegraphics[width=1\linewidth]{images/model_danych/uml}
	\caption{Diagram UML komponentów}
	\label{fig:uml}
\end{figure}

NAPISAĆ CZYM SĄ KOMPONENTY!

\section*{Diagram przypadków użycia}
\addcontentsline{toc}{section}{Diagram przypadków użycia}

SKOMPRESOWAĆ ODSTĘPY + NUMERKI ZAMIAST LITER W PODLIŚCIE! + WCIĘCIE PRZYPADEK UŻYCIA

\textbf{Przypadek użycia:} Rejestracja 

\textbf{Aktor:} Gość

\textbf{Opis:} Przypadek użycia "Rejestracja" umożliwia nowym użytkownikom stworzenie konta w systemie. Rejestracja wymaga podania informacji, takich jak nazwę, adres e-mail, dzień urodzenia oraz hasło. Umożliwia to dostęp do pełnych funkcji systemu, takich jak logowanie, przeglądanie zawartości i zarządzanie swoim profilem.

\textbf{Warunki wstępne:} Gość odwiedza ekran rejestracji. Gość nie posiada konta w systemie.

\textbf{Przebieg:}

\begin{enumerate}
	\item Gość przechodzi na ekran rejestracji.
	\item System wyświetla formularz rejestracyjny.
	\item Gość wprowadza nazwę użytkownika, adres e-mail, hasło.
	\item System sprawdza poprawność wprowadzonych danych.
	\begin{enumerate}
		\item Wprowadzone dane są poprawne.
		\begin{enumerate}
			\item Użytkownik zostaj dodany do bazy danych i autmatycznie przeniesiony do ekranu głównego.
		\end{enumerate}
		\item Wprowadzone dane są niepoprawne.
		\begin{enumerate}
			\item Zostaje wyświetlony komunikat dla poszczególnych pól, że podane dane są niepoprawne. Użytkownik nie zostaje dodany do bazy danych.
		\end{enumerate}
	\end{enumerate}
\end{enumerate}


\textbf{Przypadek użycia:} Logowanie

\textbf{Aktor:} Gość

\textbf{Opis:} Przypadek użycia "Logowanie" umożliwia użytkownikom dostęp do pełnych funkcji systemu. Aby zalogować się do systemu należy podać adres e-mail i hasło.

\textbf{Warunki wstępne:} Gość uruchamia aplikację. Użytkownik niezalogowany, nieposiadający konta w serwisie.

\textbf{Przebieg:}

1. Gość uruchamia aplikację. \\
2. System wyświetla formularz logowania. \\
3. Gość wprowadza adres e-mail oraz hasło. \\
4. System sprawdza poprawność wprowadzonych danych. \\
	4.1. Wprowadzone dane są poprawne. \\
		4.1.1. Gość zostaje zalogowany do sytemu. \\
	4.2. Wprowadzon dane są niepoprawne.\\
		4.2.1. Zostaje wyświetlony komunikat, że podane dane są niepoprawne. Gość nie zostaje zalogowany do systemu.\\
		

\textbf{Przypadek użycia:} Logowanie za pomocą konta Google

\textbf{Aktor:} Gość

\textbf{Opis:} Przypadek użycia "Logowanie za pomocą konta Google" pozwala na utworzenia konta za pomocą social mediów.

\textbf{Warunki wstępne:} Gość uruchamia aplikację. Użytkownik niezalogowany, nieposiadający konta w serwisie.

\textbf{Przebieg:}

1. Gość uruchamia aplikację. \\
2. System wyświetla formularz logowania. \\
3. Gość klika w przycisk "Google".\\
4. Zostaje przeniesiony doformularza Google w którym podaje login i hasło.\\
5. Podane dane są prawidłowe. Użytkownik zostaj dodany do bazy danych i autmatycznie przeniesiony do ekranu głównego.\\


\textbf{Przypadek użycia:} Sprawdzenie horoskopu

\textbf{Aktor:} Klient

\textbf{Opis:} Przypadek użycia "Sprawdzenie horoskopu" umożliwia użytkownikom dostęp do
codziennych horoskopów dedykowany dla ich znaku zodiaku.

\textbf{Warunki wstępne:} Klient posiada konto w systemie oraz jest zalogowany. Użytkownik nie posiada przydzielonego znaku zodiaku.

\textbf{Przebieg:}

1. Klient wybiera z dolnej nawigacji zakładkę Horocop.\\
2. Zostają wyświetlone dwa, małe formularze. Użytkownik może wybrać z listy swój znak, a jeśli nie jest pewny może w drugim formularzu podać dzień i miesiąc urodzenia, a na dole zostanie wyświetlony komunikat o znaku zodiaku dla tej daty. \\
3. Znak zodiaku zostaje zapisany do bazy danych. \\
4. Po odświerzeniu strony zostaje wyświetlona nazwa znaku zodiaku, ilustracja oraz horoskop na dany dzień.\\


\textbf{Przypadek użycia:} Dodawanie urodzin do kalendarza

\textbf{Aktor:} Klient

\textbf{Opis:} W zakładce "Year" użytkownik może dodać urodziny do kalendarza, przejść do ekranu edycji listy urodzin oraz wyświetlić zaznaczone daty na kalendarzu. 

\textbf{Warunki wstępne:} Klient posiada konto w systemie oraz jest zalogowany.

\textbf{Przebieg:} 
1. Klient wybiera z dolnej nawigacji zakładkę Year.\\
2. Wprowadza w formularzu: dzień, miesiąc oraz imię solenizant.
3. System sprawdza poprawność wprowadzonych danych. \\
3.1. Wprowadzone dane są poprawne. \\
3.1.1. Urodziny zostają dodane do bazy danych oraz wyświeietlone na kalendarzu. \\
4.2. Wprowadzon dane są niepoprawne.\\
4.2.1. Zostaje wyświetlony komunikat, że podane dane są niepoprawne. Urodziny nie zostają zapisane w bazie danych. \\

\textbf{Przypadek użycia:} Edytowanie listy urodzin

\textbf{Aktor:} Klient

\textbf{Opis:} Na ekranie wyświetla się tabelą z listą urodzin użytkownika. W każdym wierszu obok imienia znajduje się ikona kosza.

\textbf{Warunki wstępne:} Klient znajduje się w zakładce "Year" oraz wyświetla mu się portal z tabelą. Dodatkowo posiada zapisane urodziny w bazie danych.

\textbf{Przebieg:} 
1. Klient wybiera z dolnej nawigacji zakładkę Year.\\
2. Klika w przycisk z ikoną ołówka, poczym zostaje wyświetlony portal.\\
3. Użytkownik klika w ikonę kosza. \\
4. Urodziny zostają usuniętę z bazy danych, a tabla zostaje automatycznie odświeżona.\\

\textbf{Przypadek użycia:} Dodawanie celów na dany miesiąc

\textbf{Aktor:} Klient

\textbf{Opis:} Na ekraniem głównym użytkownik może dodać cele na dany miesiąc.

\textbf{Warunki wstępne:} Klient posiada konto w systemie oraz jest zalogowany.

\textbf{Przebieg:} 1. Klient został pomyślnie zalogowany do systemu.\\
2. Użytkownik wpisuje w polu cel na aktualny miesiąc, a następnie klika w ikone plusa. Cel został zapisany w bazie danych.\\
3. Przycisk zmienia symbol z plusa na ołówek. Użytkownik teraz może edytować swój cel.\\
4. Gdy cel zostanie osiągnięty klient może zaznaczyć checbox, który przekreśli tekst w inpucie, a w bazie danych pole "state" zmieni swoją wartość na true.\\ 



\textbf{Przypadek użycia:} Wykreślanie zadań z listy "to do" na dany tydzień 

\textbf{Aktor:} Klient

\textbf{Opis:} W zakładce "Week" użytkownik może dodać zobaczyć listę zadań do zrobienia na dany tydzień, ma możliwośc wykreślenia zadania po jego wykonaniu. Dodatkowo może wyznaczyć cele na ten tydzień oraz dodać nowe wydarzenia oraz zadania. 

\textbf{Warunki wstępne:} Klient posiada konto w systemie, jest zalogowany oraz ma rozpisane zadania na konretne dni.

\textbf{Przebieg:} 
1. Klient wybiera z dolnej nawigacji zakładkę "Week". Na dany dzień są wpisane konkretne zadania.\\
2. Użytkownik po wykonaniu zadania klika w checbox. \\
3. Tekst w polu zostaje przekreślony, a status w bazie danych zostaje zmieniony na "true".\\


\textbf{Przypadek użycia:} Dodawanie zadań oraz wydarzeń do kalendarza dział na podobnej zasadzie jak przypadek użycia "Edytowanie listy urodzin" z tym, że tutaj w formularzu nie ma pola "Month", a w tabeli oprócz dnia, nazwy i ikony kosza jest dodatkowe pole "status" określające, czy zadanie jest ukończone.

\textbf{Przypadek użycia:} Tworzenie rutyny pielęgnacyjnej 

\textbf{Aktor:} Klient

\textbf{Opis:} Na ekranie Routines przedstawiony jest formularz dodawania produktów dla konkretnej kategorii pielęgnacji z zwenętrznej bazy danych kosmetyków.

\textbf{Warunki wstępne:} Klient znajduje się w zakładce "SkinCare" na ekranie Routines. Posiada połączenie z internetem. 

\textbf{Przebieg:} 
1. Klient wybiera z dolnej nawigacji zakładkę "SkinCare".\\
2. Z górnej nawigacji wybiera zakładkę "Routines".\\
3. Zostają wyświetlone formularze dla poszczegónych typów pielęgnacji. \\
4. Użytkownik kilka w pole "Product" i zostaje wyświetlona lista dostępnych produktów.\\
5. Użytkownik wybiera produkty z listy.\\
6. Kosmetyki zostają dodane do tabeli poniżej. \\
7. Jeśli użytkonik przez przypadek wybierze zły produkt, może go w każdej chwili usunąć z tabeli kilikając w ikone kosza.\\


\textbf{Przypadek użycia:} Monitorowanie dziennej pielęgnacji

\textbf{Aktor:} Klient

\textbf{Opis:} W zakładce "SkinCare" użytkownik może dodać zobaczyć listę kosmetyków na dany dzień w dwóch kategoriach: Morning i Evening.

\textbf{Warunki wstępne:} Klient posiada konto w systemie, jest zalogowany oraz ma ustalony plan pielęgnacji na wieczór oraz dzień.

\textbf{Przebieg:} 
1. Klient wybiera z dolnej nawigacji zakładkę "SkinCare".\\
2. W sekcji Evening użytkownik wybiera typ pielęgnacji z menu.\\
3 Zostają wyświetlone produkty dla danego typu pielęgnacji.\\
4. Użytkownik po wykonaniu pielęgnacji zaznacza klikając w chackbox, których produktów użył.\\
5. Kafelek zmienia kolor z różowego na szary.\\


\textbf{Przypadek użycia:} Podsumowanie 

\textbf{Aktor:} Klient

\textbf{Opis:} Na ekranie Summary użytkownik może zobaczyć podsumowanie swojej pielęgnacji w danym miesiącu.

\textbf{Warunki wstępne:} Klient znajduje się w zakładce "SkinCare" na ekranie Routines. W bazie danych znajdują się informację o zrealizowanych pielęgnacjach.

\textbf{Przebieg:} 
1. Klient wybiera z dolnej nawigacji zakładkę "SkinCare".\\
2. Z górnej nawigacji wybiera zakładkę "Summary".\\
3. Zostają wyświetlone wykres kołowy oraz tabela przedstawiający w ilu procentach dana pielęgnacja była w tym miesiącu złuszczająca, regenerująca, nawilżająca lub jej nie było. \\


\chapter{Testowanie}
\addcontentsline{toc}{chapter}{Testowanie}
Napisać coś o  testach, na czym polega, rodzaje testów,
napisać coś o istqb!
Wkeić kod testów jednostkowych

Ważnym aspektem tworzenia oprogramowania jest przeprowadzanie testów, które pomagają w identyfikacji i eliminacji błędów. W ramach procesu testowania warto skupić się na różnych rodzajach testów.
Testy jednostkowe są fundamentalnym elementem procesu programistycznego. Pozwalają one na sprawdzenie indywidualnych komponentów kodu w izolacji, co ułatwia wykrywanie ewentualnych błędów. Automatyzacja testów jednostkowych przyspiesza proces weryfikacji poprawności implementacji.


\chapter{Bezpieczeństwo i archiwizacja}
Napisać coś o wielkiej konieczności zawarcia fundamentalnego zagadnienia jakim jest kopia zapasowaaaaa
Napisać coś o rodzajach ataków i środkach ostrożności które zostały zaimplementowane w ZodiaCal - tylko pisać o tym na co masz wpływ! nie że wyciek danych tylko kilkasz guziczek

Zapewnienie bezpieczeństwa aplikacji jest kluczowym elementem procesu tworzenia oprogramowania.

\section{Szyfrowania Kluczy}
W celu zabezpieczenia poufnych informacji, takich jak klucze API czy hasła, warto przechowywać je w pliku konfiguracyjnym .env. Ten plik powinien być odpowiednio zabezpieczony i nigdy nie powinien być umieszczany w repozytorium kodu źródłowego. Klucze w pliku .env powinny być szyfrowane, aby minimalizować ryzyko nieautoryzowanego dostępu.

\section{Walidacja Formularzy}
Aby uniknąć problemów związanych z wprowadzaniem błędnych danych, istotne jest przeprowadzanie walidacji formularzy po stronie klienta i serwera. Poprawna walidacja formularzy pomaga w zapobieganiu atakom typu SQL Injection oraz Cross-Site Scripting (XSS). Do realizacji tego zadania skorzystano z biblioteki formik i yarn. Aby w aplikacji sprawnie przeprowadzić walidację formularzy, wykorzystano bibliotekę Formik dostarczającą gotowe rozwiązania.

\section{Kontrola dostępu pomiędzy ekranem logowania a stroną główną}
Bezpieczeństwo aplikacji można poprawić poprzez skonfigurowanie dostępu do ekranów logowania, a strony głównej w oparciu o stan użytkownika. Tylko uprawnieni użytkownicy powinni mieć dostęp do tych funkcji, co może pomóc w zminimalizowaniu ryzyka ataków.

\section{Archiwizacja}
Regularne tworzenie kopii zapasowych bazy danych jest niezbędne w przypadku awarii systemu lub utraty danych. W aplikacji wdrożono cykliczne kopie zapasowe bazy danych za pomocą Automated Backups w technologii Google Firebase.


\chapter{Zakończenie}
Zakończenie pracy powinno zawierać ustosunkowanie się Autora do zadań wskazanych we Wstępie, a w szczególności do celu, miar i zakresu pracy oraz porównanie ich z faktycznymi wynikami pracy. Podejście takie umożliwia jasne określenie stopnia realizacji założonych celów oraz zwrócenie uwagi na wyniki osiągnięte przez Autora w ramach jego samodzielnej pracy. Ta część pracy powinna zawierać również omówienie trudności jakie wystąpiły przy realizacji pracy oraz zalet i wad przyjętego rozwiązania.


Udało się zrealizować cel pracy, którym było dostarczenie gotowego produktu. Została przeprowadzona szczegółowa analiza biznesowa, która była fundamentem do dalszego rozwoju aplikacji. Zostały zaimplementowane wszystkie funkcji z sekcji Must oraz dwa z sekcji Should. Aplikacje przetestowano z wykorzystaniem testów jednostkowych oraz dopełniono obowiązku dostarczenie odpowiedniego poziomu bezpieczeństwa aplikacji. Aby aplikacja mogła stać się pełnoprawnym produktem sklepu Play konieczne byłoby rozszerzenie planu Firebase ze względu na złożoność bazy danych dla jednego usera.

Z racji coraz częstszego wykorzystywania uczenia maszynowego, przyszłościowym krokiem byłoby wprowadzenie funkcji, która po trzech miesiącach korzystania z aplikacji i prowadzenia dzienniczka pielęgnacji proponowania harmonogram pielęgnacji na podstawie wcześniejszych danych.


\printbibliography[heading=bibintoc, title={Bibliografia}]
\end{document}
