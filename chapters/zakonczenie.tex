Zakończenie pracy powinno zawierać ustosunkowanie się Autora do zadań wskazanych we Wstępie, a w szczególności do celu, miar i zakresu pracy oraz porównanie ich z faktycznymi wynikami pracy. Podejście takie umożliwia jasne określenie stopnia realizacji założonych celów oraz zwrócenie uwagi na wyniki osiągnięte przez Autora w ramach jego samodzielnej pracy. Ta część pracy powinna zawierać również omówienie trudności jakie wystąpiły przy realizacji pracy oraz zalet i wad przyjętego rozwiązania.
