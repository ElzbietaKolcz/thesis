Zakończenie pracy powinno zawierać ustosunkowanie się Autora do zadań wskazanych we Wstępie, a w szczególności do celu, miar i zakresu pracy oraz porównanie ich z faktycznymi wynikami pracy. Podejście takie umożliwia jasne określenie stopnia realizacji założonych celów oraz zwrócenie uwagi na wyniki osiągnięte przez Autora w ramach jego samodzielnej pracy. Ta część pracy powinna zawierać również omówienie trudności jakie wystąpiły przy realizacji pracy oraz zalet i wad przyjętego rozwiązania.


Udało się zrealizować cel pracy, którym było dostarczenie gotowego produktu. Została przeprowadzona szczegółowa analiza biznesowa, która była fundamentem do dalszego rozwoju aplikacji. Zostały zaimplementowane wszystkie funkcji z sekcji Must oraz dwa z sekcji Should. Aplikacje przetestowano z wykorzystaniem testów jednostkowych oraz dopełniono obowiązku dostarczenie odpowiedniego poziomu bezpieczeństwa aplikacji. Aby aplikacja mogła stać się pełnoprawnym produktem sklepu Play konieczne byłoby rozszerzenie planu Firebase ze względu na złożoność bazy danych dla jednego usera.

Z racji coraz częstszego wykorzystywania uczenia maszynowego, przyszłościowym krokiem byłoby wprowadzenie funkcji, która po trzech miesiącach korzystania z aplikacji i prowadzenia dzienniczka pielęgnacji proponowania harmonogram pielęgnacji na podstawie wcześniejszych danych.
