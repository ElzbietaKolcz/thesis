\phantom{th} Obecnie na rynku aplikacje mobilne mogą być rozwijane przy użyciu różnych technologii, takich jak Kotlin dla platformy Android, Swift dla iOS, czy też React Native, który obsługuje obie te platformy. Wybór właściwej technologii, narzędzi i bibliotek może znacznie ułatwić proces tworzenia aplikacji, dlatego warto poświęcić dodatkowy czas na zastanowienie się, które rozwiązania najlepiej sprawdzą się w kontekście konkretnego projektu.

\section{Technologie}
Napisać czym są języki programowania, frameworki, platformy chmurowe, API\\

\phantom{Th}
Technologie stanowi szereg elementów, począwszy od języków programowania, przez frameworki ułatwiające strukturyzację projektów, po platformy umożliwiające dostęp do chmurowych usług.

\subsubsection*{\textbf{React Native}}
\phantom{th}
Ze względu na swoją kompleksowość oraz szybkość odświeżania praca została wykonana z wykorzystaniem React Native.
React Native to technologia oparta na bibliotece React,
która jest używana do tworzenia interfejsów za pomocą języka JavaScript. Biblioteka ta jest typu open source, co oznacza, że jej kod źródłowy jest udostępniony do użytku publicznego i każdy ma prawo go modyfikować według własnych potrzeb.

React Native służy do tworzenia wieloplatformowych aplikacji, znanych również jako aplikacje cross-platform,
czyli aplikacji zaprojektowanych i napisanych w taki sposób, aby działały na wielu różnych platformach i systemach operacyjnych.
Dodatkowo wykorzystuje programowanie natywne zapewniając w ten sposób zestaw komponentów niezależnych od systemu operacyjnego,
który następnie jest dynamicznie przekształcany na natywne elementy interfejsu danego systemu. Dzięki temu aplikacja jest w stanie
zachować spójność i dostosować się do  konkretnej specyfiki.

Wielką zaletą React Native jest jego szybkie odświeżanie, co przyczynia się do oszczędności czasu podczas pisania kodu.
Jest to możliwe dzięki wykorzystaniu JavaScript przez React Native \cite{reactnative}.

\subsubsection*{\textbf{Expo / Expo Go}}
\phantom{Th}
Expo to platforma typu open source dla aplikacji działających natywnie na różnych systemach operacyjnych.
Expo umożliwia testowania aplikacji na różnych urządzeniach mobilnych. Aby korzystać z Expo wymagane jest programowanie w JavaScript/TypeScript
oraz posiadanie aplikacji na urządzeniu mobilnym, które można pobrać z Sklepu Play w przypadku Androida lub z App Store w przypadku iOS
\cite{expogo}.

\subsubsection*{\textbf{Google Firebase}}
\phantom{th}
Napisać o  Cloud Firestore,  Authentication, Storage.

\subsubsection*{\textbf{Redux/Redux-Toolkit}}
\phantom{th}
Napisać tutaj też o RTK-Query!

\subsubsection*{\textbf{Git / Github}}
\phantom{Th}

\section{Biblioteki}
Napisać czym są biblioteki
\subsubsection*{\textbf{Tailwind React Native Classnames - twrnc}}
\phantom{th}

\subsubsection*{\textbf{React Native Calendars}}
\phantom{th}

\subsubsection*{\textbf{React Native Paper}}
\phantom{th}

\subsubsection*{\textbf{Formik / Yup}}
\phantom{th}

\section{Narzędzia}
\phantom{Th}
Oprócz freamworków, czy bibliotek istotnymi narzędziami w procesie tworzenia aplikacji są między innymi, edytory kodu oraz platformy internetowe, które umożliwiają tworzenie interfejsów graficzny jeszcze przed implementacją kod, lub pozwalające w schludny sposób reprezentować diagramy oraz schematy projektu.
\subsubsection*{\textbf{Visual Studio Code}}
\phantom{Th}

\subsubsection*{\textbf{Figma}}
\phantom{Th}

\subsubsection*{\textbf{Lucid}}
\phantom{Th}