\addcontentsline{toc}{chapter}{Przegląd technologiczny}
Obecnie aplikacje mobilne mogą być tworzone przy użyciu różnych technologii, na
przykład takich jak Kotlin dla platformy Android, Swift dla iOS, czy też React Native, który
obsługuje obie te platformy. Wybór właściwej technologii, narzędzi i bibliotek może znacznie
ułatwić proces tworzenia aplikacji, dlatego należy poświęcić dodatkowy czas nad
zastanowieniem się, które rozwiązania najlepiej sprawdzą się w kontekście konkretnego
projektu.

\section*{Technologie}
\addcontentsline{toc}{section}{Technologie}
Technologia stanowi szereg elementów, począwszy od języków programowania, przez
frameworki, czyli rodzaj struktury lub szkieletu, który zapewnia gotowe rozwiązania, abstrakcje i narzędzia do rozwoju interfejsu \cite{framework}, po platformy umożliwiające dostęp do chmurowych usług. To wszechstronne podejście umożliwia efektywną budowę, utrzymanie i rozwijanie projektu.
\subsubsection*{\textbf{JavaScript}}
Dopisać w oparciu o ksiązki(mam jedną na półce)

\subsubsection*{\textbf{React Native}}

Ze względu na swoją elastyczność oraz szybkość odświeżania praca została wykonana
w React Native. Jest to framework oparty na React, który można rozbudować o dodatkowe
funkcje. Z kolei React jest frameworkiem języka JavaScript, który umożliwia
implementację złożonych operacji związanych między innymi z interfejsem użytkownika na
stronach internetowych. React jest biblioteką typu open source \cite{opensource}, co oznacza, że jej kod
źródłowy jest udostępniony do użytku publicznego i każdy ma prawo go modyfikować według
własnych potrzeb. Tę bibliotekę wykorzystuje się głównie do tworzenia interfejsów ze względu na możliwość zarządzania stanem z poziomu komponentu. To oznacza, że poszczególne
elementy interfejsu, zwane komponentami, są zaprojektowane tak, aby miały kontrolę nad
swoim własnym stanem, czyli informacjami, które wpływają na ich wygląd i zachowanie. Dzięki temu podejściu, gdy stan komponentu ulega zmianie, interfejs dynamicznie reaguje, co pozwala na płynne i natychmiastowe aktualizacje interfejsu użytkownika \cite{javascripteverywhere}.

React Native służy do tworzenia wieloplatformowych aplikacji, znanych również jako cross-platform mobile applications, czyli aplikacji zaprojektowanych w taki sposób, aby działały na wielu różnych systemach operacyjnych.. Aplikacje napisane w React Native to na przykład Facebook, Instagram, Skype, czy chociażby Pinterest \cite{javascripteverywhere}.

Obiekt Process.env jest globalnym obiektem Node, a zmienne są przekazywane w postaci ciągów znaków \cite{env}. Żeby skorzystać z tego rozwiązania, należy stworzyć plik .env w którym zostaną napisane stałe przechowujące klucze API lub adresy URL w postaci tekstu jawnego, zrozumiałego dla człowieka (Rys.\ref{fig:Env}). Z racji tego, że są to zmienne globalne można w dowolnym pliku się do nich odwoływać za pomocą słowa kluczowego process.env.VAR\_NAME.

\subsubsection*{\textbf{Expo / Expo Go}}
Expo to kompleksowa platforma typu open source, zaprojektowana do tworzenia aplikacji mobilnych, które działają natywnie na różnych systemach operacyjnych. Oferuje narzędzia i usługi, które ułatwiają uruchamianie i rozwijanie projektów iOS oraz Android z użyciem React Native \cite{javascripteverywhere}. Platforma ta ułatwia procesy uruchamiania i rozwijania projektów poprzez dostarczenie narzędzi w postaci Expo CLI (Command-Line Interface) oraz aplikacji mobilnej Expo Go, na której można podglądać efekty prac na bieżąco na urządzeniu mobilnym. Aby korzystać z Expo Go wymagane jest programowanie w JavaScript/TypeScript, posiadanie aplikacji na urządzeniu mobilnym, które można pobrać z Sklepu Play w przypadku Androida lub z App Store w przypadku iOS oraz połączenie komputera, jak i telefonu w tej samej sieci Wi-Fi \cite{expogo}.
Dodatkowo wykorzystuje odpowiednie narzędzia zapewniające zestaw komponentów niezależnych od systemu operacyjnego, które następnie przekształcają kod źródłowy na natywne elementy interfejsu danej platformy. Dzięki temu aplikacja jest w stanie zachować spójność i dostosować się do konkretnej platformy \cite{reactnative}.



Redux-Toolkit powstało w odpowiedzi na reakcje użytkowników Reduxa, którzy skarżyli się na konieczność instalowania wielu dodatkowych pakietów oraz powtarzania tych samych fragmentów kodu w kilku miejscach. Redux-Toolkit to zestaw narzędzi pozwalający na uproszenie kod aplikacji. Jednym z takich narzędzi jest RTK Query pozwalający na obsługę API. API, czyli Application Programming Interface to zestaw zdefiniowanych reguł pozwalających na komunikację pomiędzy dwoma odrębnymi aplikacjami \cite{api}. RTK Query umożliwia zdefiniowanie zestawu punktów końcowych oraz obróbkę i wyświetlenie danych z zewnętrznego serwera \cite{reduxtoolkit}.

\section*{Biblioteki JavaScript}
\addcontentsline{toc}{section}{Biblioteki}
Biblioteki reprezentują zorganizowany zbiór funkcji, procedur lub klas, które zostały
precyzyjnie zaprojektowane w celu umożliwienia integracji i ponownego wykorzystania
w ramach projektu programistycznego. Ich istotnym celem jest eliminacja konieczności
powtarzalnej implementacji rutynowych zadań programistycznych poprzez dostarczenie
gotowych do użycia komponentów.

\subsubsection*{\textbf{Redux/Redux-Toolkit}}
Jak zostało napisane wcześniej, React Native wykorzystuje model projektowania oparty
na komponentach, które przechowują informacje o swoim wyglądzie i stanie. W złożonych
projektach występuje wiele komponentów, co za tym idzie, występuje wiele niezależnych
stanów, które trzeba obsłużyć. W tym celu najlepiej sprawdzi się Redux, który umożliwia
zarządzanie stanem z jednego miejsca. Ponadto dostarcza gotową logikę obsługi
otrzymywania, zwracania oraz aktualizację stanu, dzięki czemu w kodzie nie występują
niepotrzebne powtórzenia oraz implementuje wewnętrznie wiele optymalizacji wydajności,
dzięki którym komponenty odświeżają się tylko wtedy gdy faktycznie jest to konieczne \cite{learningredux}, \cite{reactredux}.

\subsubsection*{\textbf{Tailwind React Native Classnames - twrnc}}
Tailwind React Native Classnames to bibliotek wykorzystująca Tailwind CSS jako języka skryptowego do tworzenia uniwersalnego systemu styli dla aplikacji React Native.
Natomiast Tailwind CSS to framework służący do projektowania interfejsów użytkownika, bazujący na klasach CSS.
Dostarcza zestaw predefiniowanych klas, które reprezentują różne style, kolory, marginesy, czy wypełnienia dzięki czemu interfejsy są konsekwentne i spójne.

Choć React Native nie obsługuje tradycyjnych stylów CSS, narzędzie Tailwind React
Native Classnames pozwala na przekształcanie klas CSS zapisanych w stylach Tailwind na
odpowiednie style i komponenty dostępne w React Native \cite{nativewind}.

\subsubsection*{\textbf{React Native Calendars / React Native Paper}}
React Native Calendars stanowi kluczą bibliotekę dla projektu ZodiaCal, ponieważ dostarcza gotowy komponent kalendarza miesięcznego oraz tak zwanej Agendy, czyli widoku kalendarza na dany tydzień. Ponadto pozwala na zaznaczenie w kalendarzu wydarzeń okresowych oraz na wyświetlaniu wiele znaczników różnych kategorii dla danego dnia.

Z kolei biblioteka React Native Paper dostarcza gotowe komponenty takie jak przyciski, czy pola tekstowe zgodne z Google Material Design, czyli zasadami projektowania stworzonego przez Google w oparciu o innowacyjne technologie. Powstały one po to, aby doświadczenia użytkowników były względnie podobne niezależnie od aplikacji z której korzystają \cite{uxui}.

\subsubsection*{\textbf{Formik / yup}}

\section*{\textbf{Usługi chmurowe}}
Usługi chmurowe to nowoczesne rozwiązanie oferujące dostęp do zasobów obliczeniowych, przestrzeni dyskowej lub środowisk programistycznych. Jest pewnego rodzaju magazynem cyfrowym umożliwiającym dostęp do plików z dowolnego urządzenia, pod warunkiem, że jest ono połączone z internetem \cite{cloud}.

\subsection*{\textbf{Firebase}}
Dopisać co to jest OAuth2

Takim rozwiązaniem jest Google Firebase, które oferuje ogromne możliwości, między innymi tworzenie warstwy serwerowej, bez konieczności zarządzania serwerem, bezproblemowe skalowanie, narzędzia obsługujące uwierzytelniania użytkowników, a także pracę z uczeniem maszynowym \cite{firebase}.

Firebase jest nierelacyjną bazą danych w modelu dokumentowym. Bazy NoSQL powstały w odpowiedzi na problemy jakie generowały relacyjne bazy danych. Główną ich wadą jest utrudniona obsługa danych w czasie rzeczywistym. Oprócz tego występuje problem współbieżności, gdy wielu użytkowniku pracuje na tych samych danych; integracji, czyli zależność struktur dla każdej z aplikacji; impedancji polegający na niezgodności modelu danych w bazie z modelem danych w aplikacji oraz skalowania, które odbywa się tylko „górę”. Bazy nierelacyjne są remedium na tego typu problemy. Jednym z typów baz NoSQL jest model dokumentowy, w którym dokumenty są samoopisującymi hierarchicznymi strukturami drzewiastymi, które mogą się składać z wartości złożonych lub wartości skalarnych. W takim modelu kluczem może być dowolny ciąg znaków, a wartością jest dokumentem w określonym formacie \cite{nosql}.

Jako bazę danych projektu wykorzystano usługę Cloud Firestore, która pozwala na aktualizacje w czasie rzeczywistym. Model danych tego rozwiązania pozwala obsługiwać elastyczne, hierarchiczne struktury danych. Organizacja wpisów polega na tworzeniu dokumentów, a w nich kolekcji tworząc tym samym skalowanie w „bok”. Dodatkowo umożliwia wysyłanie zapytań do pobrania pojedynczych dokumentów lub do wszystkich dokumentów w kolekcji, tworzenie wielu filtrów oraz sortowanie rekordów \cite{storage}. W celu efektywnego zarządzania zasobami multimedialnymi w projekcie, skorzystano z usługi Cloud Storage do przechowywania ilustracji.

Aby zabezpieczyć proces uwierzytelniania użytkownika, zdecydowano się skorzystać z Firebase Authentication, które nie tylko oferuje szeroki zakres metod uwierzytelniania, ale także wykorzystuje kompleksowy zestaw narzędzi SDK(Software Development Kit) i gotowych bibliotek interfejsu użytkownika. Dzięki temu narzędziu możliwe jest uwierzytelnianie za pomocą haseł, numerów telefonów, mediów społecznościowych, a także uwierzytelnianie wieloskładnikowe, co zapewnia elastyczność i bezpieczeństwo procesu logowania. Dodatkowo, Firebase Authentication obsługuje funkcję odzyskiwania konta, co stanowi istotny element kompleksowej strategii zarządzania dostępem użytkowników \cite{authenticationase}.


\section*{\textbf{System kontroli wersji}}
Co to jest system kontroli wersji, jak dzialaja ogólnie i po co są
\subsection*{\textbf{Git}}
 Pro Git book, written by Scott Chacon and Ben Straub and published by Apress,
dlaczego jest w zodiacal, czym jest git, jak dziala itp
Git, czyli System Kontroli Wersji, odgrywa kluczową rolę jako niezbędne narzędzie
wspomagające pracę deweloperów. Jego głównym celem jest umożliwienie precyzyjnej
kontroli postępu projektu poprzez rejestrowanie zmian w kodzie źródłowym. Git polega na
nadpisywaniu bieżącej wersji kodu, przy jednoczesnym zachowaniu dostępu do poprzednich
wersji \cite{git}. Deweloperzy korzystając z tego rozwiązania są zabezpieczeni przed potencjalnymi błędami wprowadzonymi podczas edycji kodu, gdyż mogą sprawnie przywrócić wcześniejsze, działające wersje projektu.
