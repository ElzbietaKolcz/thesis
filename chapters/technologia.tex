Napisać o systemie kontroli wersji, co to jest Api, o Expo
Korzystać z dokumentacji!!\\

\section{Tehnologie}

\subsubsection*{\textbf{React Native}}
\phantom{Th}
React Native to technologia oparta na bibliotece React, 
która jest używana do tworzenia interfejsów za pomocą języka JavaScript. 
Biblioteka ta jest typu open source, co oznacza, że jej kod źródłowy jest udostępniony 
do użytku publicznego i każdy ma prawo go modyfikować według własnych potrzeb.\\

React Native służy do tworzenia wieloplatformowych aplikacji, znanych również jako aplikacje cross-platform, 
czyli aplikacji zaprojektowanych i napisanych w taki sposób, aby działały na wielu różnych platformach i systemach operacyjnych. 
Dodatkowo wykorzystuje programowanie natywne zapewniając w ten sposób zestaw komponentów niezależnych od systemu operacyjnego, 
który następnie jest dynamicznie przekształcany na natywne elementy interfejsu danego systemu. Dzięki temu aplikacja jest w stanie 
zachować spójność i dostosować się do  konkretnej specyfiki.\\

Wielką zaletą React Native jest jego szybkie odświeżanie, co przyczynia się do oszczędności czasu podczas pisania kodu. 
Jest to możliwe dzięki wykorzystaniu JavaScript przez React Native \cite{reactnative} .\\


\subsubsection*{\textbf{Expo / Expo Go}}
\phantom{Th}
Expo to platforma typu open source dla aplikacji działających natywnie na różnych systemach operacyjnych. 
Expo umożliwia testowania aplikacji na różnych urządzeniach mobilnych. Aby korzystać z Expo wymagane jest pisanie kodu w JavaScript/TypeScript 
oraz posiadanie aplikacji na urządeniu mobilnym, które można pobrać z Sklepu Play w przypadku Androida lub z App Store w przypadku iOS'a 
\cite{expogo} .

\subsubsection*{\textbf{Google Firebase}}
\phantom{Th}

\subsubsection*{\textbf{Redux/redux-toolkit}}
\phantom{Th}

\subsubsection*{\textbf{RTK-query}}
\phantom{Th}


\section{Biblioteki}
\subsubsection*{\textbf{Tailwind React Native Classnames - twrnc}}
\phantom{Th}

\subsubsection*{\textbf{React Native Calendars}}
\phantom{Th}

\subsubsection*{\textbf{React Native Paper}}
\phantom{Th}

\subsubsection*{\textbf{React Native Vector Icons}}
\phantom{Th}

\subsubsection*{\textbf{React Redux}}
\phantom{Th}

\subsubsection*{\textbf{Formik}}
\phantom{Th}

\subsubsection*{\textbf{Yup}}
\phantom{Th}


\section{Narzędzia}
\subsubsection*{\textbf{Github}}
\phantom{Th}

\subsubsection*{\textbf{Visual Studio Code}}
\phantom{Th}

\subsubsection*{\textbf{Figma}}
\phantom{Th}

\subsubsection*{\textbf{Lucid}}
\phantom{Th}