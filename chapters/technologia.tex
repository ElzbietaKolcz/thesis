Napisać o systemie kontroli wersji, co to jest Api, o Expo
Korzystać z dokumentacji!!\\

W poniższym rozdziale zostały przedstawione kluczowe technologie potrzebne do wykonania aplikacji.\\
\phantom{Th} 

% \section{React Native}\\


\cite{reactnative} React Native to technologia oparta na bibliotece React, która jest używana do tworzenia interfejsów za pomocą języka JavaScript.  Biblioteka ta jest typu open source, co oznacza, że jej kod źródłowy jest udostępniony do użytku publicznego i każdy ma prawo go modyfikować według własnych potrzeb. \\
 
React Native służy do tworzenia wieloplatformowych aplikacji, znanych również jako aplikacje cross-platform, czyli aplikacji zaprojektowanych i napisanych w taki sposób, aby działały na wielu różnych platformach i systemach operacyjnych. Dodatkowo wykorzystuje programowanie natywne zapewniając w ten sposób zestaw komponentów niezależnych od systemu operacyjnego, który następnie jest dynamicznie przekształcany na natywne elementy interfejsu danego systemu. Dzięki temu aplikacja jest w stanie zachować spójność i dostosować się do  konkretnej specyfiki.\\

Wielką zaletą React Native jest jego szybkie odświeżanie, co przyczynia się do oszczędności czasu podczas pisania kodu. Jest to możliwe dzięki wykorzystaniu JavaScript przez React Native.\\


% \section{NativeWind}\\

\cite{nativewind} NativeWind wykorzystuje Tailwind CSS jako język skryptowy do tworzenia uniwersalnego systemu styli dla React Native. Z kolei Tailwind CSS to framework do tworzenia interfejsów użytkownika, który opiera się na klasach CSS, ułatwiając tym samym stylizację aplikacji. Framework w tym kontekście to rodzaj struktury lub szkieletu, który zapewnia gotowe rozwiązania, abstrakcje i narzędzia do rozwoju interfejsu. Tailwind CSS dostarcza zestaw predefiniowanych klas CSS, które reprezentują różne style, kolory, marginesy, wypełnienia dzięki czemu interfejs jest konsekwentny i spójny.  \\

Choć React Native nie obsługuje tradycyjnych stylów CSS, narzędzie NativeWind pozwala na przekształcanie klas CSS zapisanych w stylach Tailwind na odpowiednie style i komponenty dostępne w React Native, umożliwiając wykorzystanie tych klas w aplikacjach mobilnych.\\

% \section{Expo Go}\\
\cite{expogo} Expo to platforma typu open source dla aplikacji działających natywnie na różnych systemach operacyjnych. Expo umożliwia testowania aplikacji na różnych urządzeniach mobilnych. Aby korzystać z Expo wymagane jest pisanie kodu w JavaScript/TypeScript oraz posiadanie aplikacji na urządeniu mobilnym, które można pobrać z Sklepu Play w przypadku Androida lub z App Store w przypadku iOS'a\\


% \section{Google Firebase}\\
Choć React Native nie obsługuje tradycyjnych stylów CSS, narzędzie NativeWind pozwala na przekształcanie klas CSS zapisanych w stylach Tailwind na odpowiednie style i komponenty dostępne w React Native, umożliwiając wykorzystanie tych klas w aplikacjach mobilnych.\\


% \section{Postman}\\
Choć React Native nie obsługuje tradycyjnych stylów CSS, narzędzie NativeWind pozwala na przekształcanie klas CSS zapisanych w stylach Tailwind na odpowiednie style i komponenty dostępne w React Native, umożliwiając wykorzystanie tych klas w aplikacjach mobilnych.\\


% \section{Visual Studio Code}\\
Choć React Native nie obsługuje tradycyjnych stylów CSS, narzędzie NativeWind pozwala na przekształcanie klas CSS zapisanych w stylach Tailwind na odpowiednie style i komponenty dostępne w React Native, umożliwiając wykorzystanie tych klas w aplikacjach mobilnych.\\


% \section{Github}\\
Choć React Native nie obsługuje tradycyjnych stylów CSS, narzędzie NativeWind pozwala na przekształcanie klas CSS zapisanych w stylach Tailwind na odpowiednie style i komponenty dostępne w React Native, umożliwiając wykorzystanie tych klas w aplikacjach mobilnych.\\


% \section{Figma}\\
Choć React Native nie obsługuje tradycyjnych stylów CSS, narzędzie NativeWind pozwala na przekształcanie klas CSS zapisanych w stylach Tailwind na odpowiednie style i komponenty dostępne w React Native, umożliwiając wykorzystanie tych klas w aplikacjach mobilnych.\\

