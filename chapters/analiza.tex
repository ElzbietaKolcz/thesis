\section{Analiza konkurencyjnych rozwiązań}

\phantom{Th} 
ZodiaCal to aplikacja, która integruje funkcje kalendarza biurowego z możliwościami prowadzenia dzienniczka pielęgnacji cery. \\

Jak dotąd rynek rozdzielał te aplikacje na osobne produkty ze względu na złożoność funkcji obu rozwiązań, co może budzić w użytkownikach przeciążenia informacyjnego  i prowadzić do frustracji. ZodiaCal korzysta z podstawowych funkcji obu tych założeń, dzięki czemu już od pierwszego użycia jest prosta i przyjemna w obsłudze. 

\section{Przykładowe aplikacje}
Wpleść w tekst przypisy dolne z worda\\

\phantom{Th} 
Na rynku znajduje się wiele narzędzi do organizacji czasu, ale \textbf{kalendarz Google} pozostaje niekwestionowanym liderem. Jest to nieodłączny element strategii firmowych, ułatwiający harmonogramowanie spotkań, śledzenie projektów i zachowanie kontroli nad terminami. Jest to narzędzie idealne do precyzyjnych zadań wymierzonych co do minuty, jednak nie każdy potrzebuje wszędzie rozpiski godzinowej, lokalizacji wydarzenia, czy linku do rozmowy wideo. ZodiaCal jest bardziej skierowanych do pojedynczych jednostek, niż do prac zespołowych.\\

Bardzo podobną platformą do prowadzenia kalendarza jest usługa firmy Proton \textbf{ProtonCalendar}, jednak i ona jest ukierunkowana na zespoły pracujące w firmach, z tą różnicą, iż proton specjalizuje się w zapewnianiu prywatności i bezpieczeństwa użytkowników.\\

\textbf{FeelinMySkin Skincare Routine} to aplikacja umożliwiająca prowadzenie dzienniczka pielęgnacyjnego, która cieszy się największą popularnością* w sklepie Google Play . Aplikacja pozwala na dokładne zapisywanie każdego kroku w pielęgnacji, uwzględniając konkretne produkty danej marki, oraz oferuje możliwość interakcji z społecznością zainteresowaną tym tematem. Jest to narzędzie bardzo dokładne. ZodiaCal jest aplikacją skierowaną do osób chcących dopiero rozpocząć swoją przygodę z organizację czasu i pielęgnacji. 

\section{Analiza SWOT}
Napisać tutaj coś o analizie SWOT

\phantom{Th} 

\textbf{Mocne strony (Strengths):}\\
•	Prostota: Aplikacja łączy w sobie wyłącznie najpotrzebniejsze funkcje typowego, biurowego kalendarza i dzienniczka pielęgnacji, przez co jest bardziej przystępna dla początkujących.\\
•	Spersonalizowana pielęgnacja: Możliwość dostosowywania rutyn pielęgnacyjnych do konkretnych potrzeb użytkownika.\\
•	Powiadomienia i przypomnienia: Ułatwia utrzymanie regularności w dbaniu o cerę.\\
•	Uczenie maszynowe: Propozycje pielęgnacji na podstawie wcześniejszych danych.\\
•	Wieloplatformowość: Aplikacja działa na wielu platformach dzięki czemu można dotrzeć do szerszej grupy użytkowników.\\

\textbf{Słabe strony (Weaknesses):}\\
•	Brak bazy danych produktów: W aplikacji nie ma możliwości zaciągnięcia z bazy danych kosmetyków, których użytkownik używa podczas pielęgnacji.\\
•	Brak synchronizacji z pocztą: Jeśli użytkowni otrzyma maila odnośnie jakiegoś wydarzenia, nie doda się ono automatycznie do kalendarza.\\

\textbf{Szanse (Opportunities):}\\
•	Rynek pielęgnacji cery: W ostatnich czasach zainteresowanie świadomą pielęgnacją skóry wzrosło co stwarza możliwość przyciągnięcia nowych użytkowników.\\
•	Wprowadzenie płatnej wersji z dodatkowymi funkcjami: Możliwość wprowadzenia płatnej wersji aplikacji z rozszerzonymi funkcjami dla zaawansowanych użytkowników.\\

\textbf{Zagrożenia (Threats):}\\
•	Prywatność i bezpieczeństwo danych: Konieczność zapewnienia wysokiego poziomu ochrony danych użytkowników, zwłaszcza w obszarze związanym z informacjami o pielęgnacji skóry.\\
•	Zmiany w regulacjach dotyczących prywatności: Zmiany przepisów dotyczących ochrony danych osobowych mogą wpłynąć na wymogi dotyczące zachowania prywatności użytkowników i mogą wymagać dostosowania polityki prywatności aplikacji.\\
