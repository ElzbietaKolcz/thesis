\phantom{Th}
W obecnych warunkach społeczno-ekonomicznych wielu studentów podczas toku swojego nauczania podejmuję pracę dorywczą.
Zarządzanie nawarstwiającymi się wydarzeniami typu egzaminy, projekty zaliczeniowe, spotkania biznesowe,
czy po prostu wyjścia ze znajomymi mogą być dla wielu osób przytłaczające.
Do tego wszystkiego dochodzi moda na prowadzenie zdrowego trybu życia, które niejako wiąże się z dbaniem o cerę poprzez skomplikowane,
rozbudowane pielęgnację z użyciem różnych produktów. Wśród młodych dorosłych zanika potrzeba prowadzenia klasycznego,
papierowego kalendarza. Wszystko odbywa się z wykorzystaniem aplikacji mobilnych, ponieważ jest to wygodne rozwiązanie,
telefon jest zawsze w zasięgu ręki, a powiadomienia o nadchodzących wydarzeniach wielokrotnie były nieocenionym wsparciem.

Stworzona na potrzeby pracy dyplomowej aplikacja stara się ułatwić użytkownikom efektywną organizację wielu wydarzeń,
a także zapewnienia wsparcia w codziennej pielęgnacji cery. Opracowane narzędzie posiada opcję automatycznego uzupełniania pól w kalendarzu
z rozróżnieniem na kategorie praca / studia co w przypadku cyklicznych wydarzeń znacząco oszczędza czas i eliminuje konieczność
wykonywania powtarzalnych czynności przez użytkownika. Dodatkowo aplikacja ma wbudowaną funkcję analizy pielęgnacji cery umożliwiającą
monitorowanie w jakim stopniu pielęgnacja w danym miesiącu była nawilżająca, złuszczająca czy regeneracyjna.
Co więcej użytkownicy mają dostęp do codziennych horoskopów dostosowanych do ich znaków zodiaku.
Jeśli użytkownik nie zna bądź nie wie w jaki sposób określić swój znak zodiaku, aplikacja po utworzeniu konta i kliknięcia w zakładkę Horoscope
pyta o dzień i miesiąc urodzenia, a następnie automatycznie przydziela znak zodiaku na podstawie podanej daty.

Celem pracy inżynierskiej jest zapewnienie wsparcia dla ambitnych młodych dorosłych, którzy dążą do maksymalnego wykorzystania życia.
