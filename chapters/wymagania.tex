\section{Analiza przedmiotowa MoSCoW}
\phantom{Th}
Analiza MoSCoW podobnie jak analiza SWOT jest akronimem. Dzieli wymagania na cztery kategorię, uwzględniające wzrastającą złożoność.
Są to kolejno M (Must have) wymagania konieczne, S (Should have) wymagania wskazane, C (Could have) wymagania opcjonalne,
W (Won't have) wymagania wykluczone. Metoda ta pomaga w ustaleniu priorytetów, które w sposób uzasadniony zapewnia kompletny
i dokładny zestaw wymagań, które aplikacja musi spełnić. Takie podejście uświadamia, które funkcje są niezbędne do funkcjonowania systemu
i zapobiega realizacji zadań opcjonalnych przed ukończeniem zadań obowiązkowych. W rezultacie umożliwia bardziej efektywne zarządzanie projektem i zaspokojenie kluczowych potrzeb użytkowników \cite{moscow},\cite{businessanalysis}.

\subsubsection*{\textbf{Must have}}
\phantom{Th}
Zbiór funkcji oznaczony jako Must have to kategorie wymagań, które system jest zobowiązany posiadać.
Podstawowym elementem aplikacji jest połączenie z bazą danych, którą należy w pierwszej kolejności skonfigurować
i podłączyć. Konieczne jest utworzenie modelu danych oraz zapewnienie niezbędnych operacji,
które umożliwią efektywne zarządzanie danymi użytkowników w systemie. Aby korzystać z aplikacji potrzebne jest konto,
co oznacza, że niezbędnym elementem jest ekran logowania i rejestracji.
Ponadto ekrany te muszą posiadać walidację formularzy przed wysłaniem ich na serwer. Zapewni to poprawność
i integralność danych wprowadzanych przez użytkowników oraz pomoże uniknąć potencjalnych błędów
i problemów związanych z danymi przechowywanymi w systemie.

Drugim fundamentem aplikacji jest funkcja kalendarza. Musi ona zapewniać możliwość dodawania, usuwania oraz edytowania wydarzeń, zadań.
Dodatkowo ważnym aspektem jest widok kalendarza na cały rok, miesiąc, tydzień i dzień.
Niezbędną funkcją jest również możliwość rejestrowania swojej pielęgnacji na dany dzień, w zależności od kategorii nawilżanie,
złuszczanie, odbudowa, przerwa. Kolejną kluczową funkcją jest wyświetlanie horoskopu dla znaku zodiaku użytkownika,
co wiąże się ze znalezieniem odpowiedniego API oraz połączenia go z aplikacją.

\subsubsection*{\textbf{Should have}}
\phantom{Th}
W kontekście sekcji Should have skupiamy się na wymaganiach, które są ważne, ale nie są absolutnie niezbędne dla funkcjonowania systemu.
Przydatną funkcją związaną z kalendarzem, z perspektywy użytkownika końcowego jest możliwość automatycznego uzupełnianie pól cyklicznych
takich jak praca/studia oraz otrzymywanie powiadomień. Z kolei rozszerzeniem funkcji pielęgnacji jest opcja związana z dodawaniem
produktów do danej pielęgnacji. Natomiast rozbudowaniem modułu horoskopu jest funkcja przydzielania znaku zodiaku
z podanej daty przez użytkownika. Wiele aplikacji umożliwia logowanie się za pomocą kont z mediów społecznościowych, dlatego warto rozważyć dodanie opcji logowania przy użyciu konta Google, aby sprostać obowiązującym standardom. To umożliwi rozszerzenie funkcjonalności systemu, dostarczając użytkownikom dodatkowych wygodnych i atrakcyjnych opcji.

\subsubsection*{\textbf{Could have}}
\phantom{Th}
Funkcje w grupie Could have to wymagania, które dobrze jest mieć, ale nie są kluczowe.
Dobrym obszarem do rozbudowy jest moduł związany z pielęgnacją.
Można rozszerzyć go o wyświetlaniu wykresów pod koniec miesiąca,
monitorowanie daty ważności używanych produktów oraz dodanie Trackerów samopoczucia, snu i nawyków.
Tracker to narzędzie pozwalające na odnotowanie aktywności w danym stopniu na konkretny dzień.
Na pewno dużym ułatwieniem dal użytkownika byłoby wprowadzenie możliwości zmiany języka oraz onboardingu zaraz po zarejestrowaniu konta.

\subsubsection*{\textbf{Won't have}}
\phantom{Th}
Wymagania w grupie Won't have mają najniższy priorytet. Określają funkcję, które w tej wersji nie zostaną dostarczone,
ale będą zawarte w kolejnej aktualizacji. Klienci często lubią edytować podstawowy interfejs pod siebie,
więc wprowadzenie personalizacji motywów na pewno korzystnie wpłynęłaby na odbiór aplikacji.
Dodatkowo coraz więcej użytkowników decyduje się na smartwatche.
Przyszłościowym krokiem byłoby umożliwienie korzystania z aplikacji również na zegarku.

\section{Wymagania funkcjonalne:}
\phantom{Th}
Podstawowymi funkcjonalnościami jest wszystko to co jest związane z obsługą
kalendarza, czyli dodawanie, usuwanie oraz edycja wydarzeń, zadań, celów czy harmonogramu
na dany dzień, miesiąc. Dodatkowo konieczne jest, aby użytkownik mógł zobaczyć widok
kalendarza na dany miesiąc, tydzień oraz rok. ZodiaCal to nie tylko kalendarz, ale również
dzienniczek pielęgnacji cery. Stąd potrzeba uwzględnienia możliwości wyboru typu pielęgnacji,
dodania, usunięcia, edytowania produktów do danej pielęgnacji oraz funkcja pozwalająca
zarejestrować pielęgnację na dany dzień. Ponadto aplikacja dostarcza codziennie
użytkownikowi horoskop na dany dzień uwzględniając podany wcześniej przez niego znak
zodiaku, lub datę urodzenia.

\section{Wymagania niefunkcjonalne:}
\phantom{Th}
Aby korzystać z aplikacji ZodiaCal, konieczne jest posiadanie telefonu z systemem
Android w wersji 5.0 lub nowszej, lub systemem iOS w wersji 13 lub nowszej. Ponieważ
aplikacja pobiera dane w czasie rzeczywistym i wykorzystuje podstawowy pakiet Firebase,
należy wziąć pod uwagę, że maksymalnie 100 urządzeń może jednocześnie synchronizować się
z bazą danych. Dodatkowo, maksymalna przepustowość danych jest zazwyczaj ograniczona do
360 MB na dzień, co reguluje polityka ograniczeń Firebase \cite{limits}.