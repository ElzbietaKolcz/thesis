Część teoretyczna, z wybranej tematyki realizowanego kierunku studiów;powinnabyć opartao przegląd literatury i praktyki produkcyjnej/usługowej naświetlającejstan wiedzy na dany temat -tzn.obejmujmowaćrozdziały pisane na podstawie literatury, której wykaz zamieszczany jest w części pracy Spis materiałów źródłowych. W tekście pracy muszą wystąpić odwołania do wszystkich pozycji zamieszczonych w wykazie literatury. Odnośniki do literaturynależy umieszczać w stopce strony. Autor pracy dyplomowej –inżynierskiejjest bezwzględnie zobowiązany do wskazywania źródeł pochodzenia informacji przedstawianych w pracy;dotyczy to również rysunków, tabel, fragmentów kodu źródłowego programów itd. Należy także podać adresy stron internetowych z datą dostępu w przypadku źródeł pochodzących z Internetu.

\phantom{Th} 

\section{Wymagania funkcjonalne}\\
Choć React Native nie obsługuje tradycyjnych stylów CSS, narzędzie NativeWind pozwala na przekształcanie klas CSS zapisanych w stylach Tailwind na odpowiednie style i komponenty dostępne w React Native, umożliwiając wykorzystanie tych klas w aplikacjach mobilnych.\\


\section{Wymagania niefunkcjonalne}\\
Choć React Native nie obsługuje tradycyjnych stylów CSS, narzędzie NativeWind pozwala na przekształcanie klas CSS zapisanych w stylach Tailwind na odpowiednie style i komponenty dostępne w React Native, umożliwiając wykorzystanie tych klas w aplikacjach mobilnych.\\


\section{Przypadki użycia}\\
Choć React Native nie obsługuje tradycyjnych stylów CSS, narzędzie NativeWind pozwala na przekształcanie klas CSS zapisanych w stylach Tailwind na odpowiednie style i komponenty dostępne w React Native, umożliwiając wykorzystanie tych klas w aplikacjach mobilnych.\\
