Część teoretyczna, z wybranej tematyki realizowanego kierunku studiów;powinnabyć opartao przegląd literatury i praktyki produkcyjnej/usługowej naświetlającejstan wiedzy na dany temat -tzn.obejmujmowaćrozdziały pisane na podstawie literatury, której wykaz zamieszczany jest w części pracy Spis materiałów źródłowych. W tekście pracy muszą wystąpić odwołania do wszystkich pozycji zamieszczonych w wykazie literatury. Odnośniki do literaturynależy umieszczać w stopce strony. Autor pracy dyplomowej –inżynierskiejjest bezwzględnie zobowiązany do wskazywania źródeł pochodzenia informacji przedstawianych w pracy;dotyczy to również rysunków, tabel, fragmentów kodu źródłowego programów itd. Należy także podać adresy stron internetowych z datą dostępu w przypadku źródeł pochodzących z Internetu.

\phantom{Th} 

\section{Wymagania funkcjonalne}\\
\begin{enumerate}
  \item Możliwość zalogowania się do konta
  \item Możliwość utworzenia konta
  \item Możliwość edycji danych konta
  \item Możliwość usunięcia konta z bazy danych
  \item Możliwość dodania, usunięcia, edytowania wydarzeń 
  \item Możliwość dodania harmonogramu studia/praca
  \item Możliwość dodania celów na dany miesiąc
  \item Możliwość dodania celów na dany tydzień
  \item Możliwość dodania zadań na dany dzień 
  \item Możliwość przejrzenia kalendarza rocznego z zaznaczonymi wydarzeniami
  \item Możliwość przejrzenia kalendarza miesięcznego
  \item Możliwość przejrzenia kalendarza tygodniowego
  \item Możliwość sprawdzenia horoskopu
  \item Możliwość wyboru typu pielęgnacji
  \item Możliwość dodania, usunięcia, edytowania produktów do danej pielęgnacji
  \item Możliwość dodania zrealizowanej pielęgnacji na dany dzień 
  \item Możliwość dodania, usunięcia, edytowania zwyczajów w habit tracker
  \item Możliwość dodania zrealizowanego zwyczaju 
  \item Możliwość sprawdzenia wykresów pielęgnacji i habit trackerami 
  \item Możliwość dodania liczby przespanych godzin
  \item Możliwość dodania obecnego stanu samopoczucia według skali 
\end{enumerate}

\section{Wymagania niefunkcjonalne}\\

\begin{enumerate}
  \item Maksymalnie 100 urządzeń może synchronizować się z bazą danych naraz.
  \item Maksymalna przepustowość danych jest ograniczona do 10 GB na miesiąc.
  \item Wersja Android 5.0 i wyżej
  \item Wersja IOS 13 i wyżej
  \item W przypadku błędów systemu, użytkownik może zgłosić błędy za pomocą formularza
\end{enumerate}


  
\section{Przypadki użycia}\\
